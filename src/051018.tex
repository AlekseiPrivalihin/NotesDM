\subsection{05.10.18}
\subsubsection{Антицепь}
A - произвольное множество, строго частично упорядоченное отношением R. $X \subseteq A$ называют антицепью, если $\forall x, y \in X \; (x, y) \not\in R(X)$
\subsubsection{Разбиение на антицепи}
Утверждение: всякое конечное частично упорядоченное множество высоты (высота множества - максимальная глубина элемента в нем) h можно разбить на h антицепей. Можно заметить, что кратчайшее расписание, существование которого конструктивно доказано в прошлой лекции, является как раз таким разбиением на антицепи.
\subsubsection{Лемма Дилуорса}
Во всяком конечном частично упорядоченном множестве A, $|A| = n$ $\forall t \in 1:n$ существует либо цепь длины $> t$, или антицепь длины $\geq \frac{n}{t}$. \\
Доказательство: предположим обратное. Тогда максимальная цепь в A имеет длину $\leq t$. Тогда A по предыдущему утверждению можно разбить на t антицепей. Пусть длина самой длинной из них равна l, $l < \frac{n}{t}$. Тогда в A не может быть больше чем $l * t$ элементов, но тогда получается, что $|A| = l * t < \frac{n}{t} * t = n$, что противоречит условию.
\subsubsection{Последовательности}
$S = 3, 2, 1, 8, 9, 4, 5, 6, 7$ - последовательность.\\
$S_1 = 2, 1, 4, 6$ - подпоследовательность.\\
$S_2 = 1, 4, 5, 6$ - возрастающая подпоследовательность.\\
$S_3 = 3, 2, 1$ - убывающая подпоследовательность.\\
$"<_S"$ - бинарное отношение на S. $a <_S b$ тогда и только тогда, когда a предшествует b в последовательности S.\\
$"\prec"$ - бинарное отношение на S. $a \prec b \Leftrightarrow \{_{a < b}^{a <_S b}$\\
Цепь в S (частично упорядоченной $\prec$) - возрастающая подпоследовательность. Антицепь - убывающая подпоследовательность.
\subsubsection{Возрастающая подпоследовательность наибольшей длины}
Алгоритм: Начнем строить разбиение последовательности на антицепи (убывающие подпоследовательности) следующим образом: проходя по последовательности слева направо, будем добавлять каждый следующий элемент последовательности в минимальный по номеру элемент разбиения, в который можем. Ничерта не понятно? Вот пример:\\
Последовательность $3, 5, 8, 9, 4, 6, 1, 2, 7, 10$. Сначала добавим 3 в первый элемент разбиения, поскольку он пуст и нам ничего не мешает. Добавить в первый элемент 5 мы не можем, т.к. $3 \prec 5$, добавляем во второй. По тем же причинам, 8 идет жить в третий элемент разбиения, 9 - в четвертый. А вот 4 можно добавить во второй элемент разбиения, поскольку $3 \prec 4$, но $5 \not\prec 4$. Дальше 6 идет в третий элемент, 1 - в первый, 2 - во второй, 7 - в четвертый, а 10 - в пятый. Получаем разбиение: $\Lambda = \{\{3, 1\}, \{5, 4, 2\}, \{8, 6\}, \{9, 7\}, \{10\}\}$.\\
Теперь построим цепь: Обозначим $|\Lambda| = h$. $\forall i \in 2:h \; \forall a \in \Lambda_i \; \exists prev(a)$ - элемент, лежащий в $\Lambda_{i - 1}$ такой, что $prev(a) \prec a$ (если нет, то по алгоритму a бы попал в $\Lambda_{i - 1}$). Таким образом, можно построить цепь $A$ длины h таким образом: $A_h$ - любой элемент $\Lambda_h$; $\forall i \in 1:(h - 1) \; A_i = prev(A_{i + 1})$. Поскольку в S существует цепь длины h, S нельзя разбить менее чем на h антицепей (иначе два элемента из этой цепи окажутся в одной антицепи по принципу Дирихле), то есть разбиение $\Lambda$ является минимальным. Поскольку разбиение $\Lambda$ минимально, нельзя построить цепь длины большей чем h (иначе, опять-таки, два элемента этой более длинной цепи попадут в один элемент разбиения $\Lambda$ по принципу Дирихле). Таким образом, мы одновременно научились строить самую длинную возрастающую подпоследовательность и минимальное разбиение на убывающие подпоследовательности.
\subsubsection{Теорема Дилуорса}
A - конечное частично упорядоченное отношением R множество. Доказать, что размер наибольшей антицепи Z равен размеру наименьшего разбиения на непересекающиеся цепи.\\
Доказательство: размер любой антицепи меньше либо равен количеству цепей в любом разбиении, так как иначе получится, что в такой антицепи, размер которой больше размера некоторого разбиения на цепи, два элемента этой антицепи будут лежать в одной цепи в разбиении.\\
Размер любой антицепи меньше либо равен размеру максимальной антицепи, количество элементов в наименьшем разбиении на цепи меньше либо равно количеству элементов в любом разбиении. Таким образом, получаем, что размер максимальной антицепи меньше либо равен количеству цепей в наименьшем разбиении. Притом получаем, что если размер какой-то антицепи равен размеру какого-то разбиения на цепи, то антицепь - максимальной длины, а разбиение - минимального размера. Осталось доказать строгое равенство.\\
База индукции: если $A = \varnothing$ или $R = \varnothing$, теорема очевидна. Если $|A| = 1$ - тоже. \\
Тогда рассмотрим случай $|A| \geq 2, R \not= \varnothing$. Тогда в A есть максимальный элемент M.\\
Индукционное предположение: для всех множеств, чья мощность меньше мощности A, теорема доказана.
Индукционный переход: $A \setminus {M}$ разбивается на цепи $c_1, \; ... \; , c_k$, и притом в $A \setminus {M}$ существует хотя бы одна антицепь размера k.\\
Каждая антицепь размера k пересекается с каждой цепью разбиения по 1 элементу (иначе 2 элемента антицепи попадут в одну цепь).\\
Определим $\forall i \in 1:k \; c_i \in C_i$ - максимальный элемент $C_i$, который лежит хотя бы в одной антицепи размера k. Докажем, что $Y = \{c1, \; ... \; , c_k\}$ - антицепь. \\
Зафиксируем для каждого $c_i$ какую-нибудь антицепь $Y_i$ размера k, которая содержит этот элемент (такие антицепи для разных $c_i$ могут пересекаться). $\forall i \not= j \in 1:k \; Y_i \cap C_j$ содержит ровно один элемент $y_{ij}$. $(y_{ij}, c_j) \in R$ в силу определения $c_j$. \\
$(c_j, c_i) \in R \Rightarrow (y, c_i) \in R$, чего быть не может, поскольку $c_i \in Y_i$ и $y_{ij} \in Y_i$, а $Y_i$ - антицепь. Значит, $(c_j, c_i) \not\in R$.\\
Аналогично рассмотрим $Y_j \cap C_i$ и получим, что $(c_i, c_j) \not\in R$.\\
Таким образом, Y - антицепь размера k. \\
Теперь вспомним, что мы выкидывали из A M. Если $\exists i \in 1:k \; (c_i, M) \in R$, то рассмотрим $X = \{x \in C_i \; : \; (x, c_i) \in R\} \cup {M}$. X, по построению, цепь.\\
$A \setminus X$ не содержит антицепи длины k, так как любая антицепь длины k должна иметь хотя бы один элемент в $C_i$, но в $C_i \setminus X$ по определению $c_i$ нет элементов, которые входят в антицепь длины k. $A \setminus X$ содержит антицепь размера k - 1, а значит, по индукционному предположению, разбивается на k - 1 цепь. Тогда добавив к этому разбиению X, получим, что A разбивается на k цепей. \\
В случае же, если $\forall i \in 1:k \; (c_i, M) \not\int R$ ($(M, c_i) \not\in R$ так как M - максимальный элемент в A), $Y \cup {M}$ - антицепь размера k + 1. Больше быть не может, так как мы добавили только один элемент. В разбиение $C_i$ должен добавиться один элемент. Не больше, так как существует разбиение $C_i \cup {M}$. Не меньше, так как размер разбиения на цепи не меньше чем размер любой антицепи, а у нас есть антицепь размера k + 1. \\
Таким образом, теорема Дилуорса доказана.
\subsubsection{Альтернативное доказательство теоремы Дилуорса}
Если $A = \varnothing$ или $R = \varnothing$, теорема очевидна. Если $|A| = 1$ - тоже. Если в R лежат только пары вида $(x, x)$ - тоже.\\
Тогда рассмотрим случай $|A| \geq 2, R \not= \varnothing, \exists x \not= y \in A \; (x, y) \in R$.\\
В таком множестве можно выбрать минимальный элемент m и максимальный M такие, что $(m, M) \in R), m \not= M$. Почему?\\
(Упражнение) Потому что рассмотрим произвольные $x \not= y \in A, (x, y) \in R$. Если y - максимальный, выберем его как M, а иначе существует $y_1$ такой, что $(y, y_1) \in R$. Будем "увеличивать" $y_i$ таким образом, пока не наткнемся на максимальный элемент $y_{max}$ (а мы наткнемся, т.к. множество конечно). При этом очевидно, что $y_{max} \not= x$, иначе у нас сломается антисимметричность R между x и y. Обозначим $M = y_{max}$. Аналогично найдем и m. \\
Теперь рассмотрим $A \setminus \{m, M\}$. Пусть Y - наибольшей длины антицепь в A, $|Y| = s$.\\
Рассмотрим два случая: \\
1 случай - наибольшая антицепь в $A \setminus \{m, M\}$ имеет длину $s - 1$ ($s - 2$ не может быть, так как тогда в Y должны входить и m, и M, а $(m, M) \in R$). Тогда по индукционному предположению (да, мы все еще живем по индукции) получается, что $A \setminus \{m, M\}$ разбивается на s - 1 цепь. Добавляем к этому разбиению цепь $\{m, M\}$ и получаем разбиение размера s для A. Разбиения размера $s - 1$ для A быть не может, так как иначе максимальная антицепь в A будет иметь размер $s - 1$.\\
2 случай - наибольшая антицепь Z в $A \setminus \{m, M\}$ имеет длину s. Тогда определим:\\
$A^+ = \{a \in A \; : \; \exists z \in Z \; (a, z) \in R\}$\\
$A^- = \{a \in A \; : \; \exists z \in Z \; (z, a) \in R\}$\\
Докажем, что $A^+ \cap A^- = Z$. $Z \subseteq A^+ \cap A^-$ в силу рефлексивности R. $A^+ \cap A^- \subseteq Z$? Предположим обратное. Тогда $\exists z_{\pm} \in (A^+ \cap A^-) \setminus Z$ такое, что $\exists z_1, z_2 \in Z$ такие, что $(z_{\pm}, z_1) \in R$ и $(z_2, z_{\pm}) \in R$, но тогда по транзитивности R получается, что $(z_1, z_2) \in R$, а Z таки антицепь.\\
Докажем, что  $A^+ \cup A^- = A$: Предположим обратное. Тогда $\exists z_{\mp} \in Z \; \forall z \in Z \; (z, z_{\mp}) \not\in R, (z_{\mp}, z) \not\in R$, но тогда $Z \cup \{z_{\mp}\}$ - цепь длины $s + 1$, что противоречит условию.\\
$m \not\in A^-$, так как m - минимальный элемент в A, и при этом $m \not\in Z$, так как Z мы строили в множестве $A \setminus \{m, M\}$, значит, $m \in A^+$\\
Аналогично, $M \in A^-$\\
$A^+$ и $A^-$ содержат Z, а значит, каждая цепь в $A^+$ и $A^-$ пересекает Z в одной точке, причем $\forall z \in Z$ цепь $C^+_z \subseteq A^+, z \in C^+_z$ заканчивается в z, а цепь $C^-_z \subseteq A^-, z \in C^-_z$ в нем начинается, а значит, A разбивается на цепи вида $C^+_z \cup C^-_z$, которых ровно s штук, поскольку каждая из этих цепей проходит через свой элемент Z.
