\subsection{28.02.19}
\subsubsection{Случайные процессы, цепи}
В целом, процесс - это изменение некоторой системы с течением времени (плавление металла, движение тела, брошенного под углом к горизонту, сон студента на паре). Но рассматривать непрерывные процессы, в которых у системы континуум состояний нам как-то не с руки, поэтому вводится некоторая математическая модель:\\
Пусть есть дискретное время $T \subseteq \mathbb{N}$, пространство состояний системы $M$. Тогда случайным процессом (с дискретным временем, другие мы рассматривать, надеюсь, не будем) называется последовательность ${\xi_n, n \in T}$, где $\forall n \in T \; Im(\xi_n) \subseteq M, \sum\limits_{s \in M} Pr\{\xi_n = s\} = 1$. Вообще говоря, $M$ может не быть дискретным, и тогда $\xi_n$ будут непрерывными случайными величинами, но у нас такого не будет, поэтому $M$ всегда не более чем счетно, и $\xi_n$ тогда будут ДСВ. Случайный процесс с дискретным временем называют также случайной цепью.\\
Траекторией длины n случайного процесса называют последовательность $r_0, \; ... \; , r_n$ такую, что $\xi_i = r_i \; \forall i \in 0:n$ - первое $n + 1$ (либо "длина" тут относится к количеству смен состояний, либо я просто криво записал) состояние, через которое прошла система.\\
Пример: случайные блуждания по точкам с целой координатой числовой прямой. Если в данный момент мы стоим в точке с координатой $x$, то с вероятностью $p$ мы переходим в точку $x + 1$, и c вероятностью $1 - p$ - в $x - 1$.\\
В этом процессе $Pr\{\xi_n = x | \xi_0 = x_0, \; ... \; \xi_{n - 1} = x_{n - 1}\} =  p$, если $x_{n - 1} + 1 = x$, $1 - p$, если $x_{n - 1} - 1 = x$ и 0 иначе.
\subsubsection{Марковские цепи}
Заметим, что в предыдущем примере распределение $\xi_n$ зависит только от значения $\xi_{n - 1}$, но не от предыдущих состояний системы и n. Иными словами, $Pr\{\xi_n = x | \xi_0 = x_0, \; ... \; , \xi_{n - 1} = x_{n - 1}\} = Pr\{\xi_n = x | \xi_{n - 1} = x_{n - 1}\}$.\\
Такие процессы, пространство состояний которых конечно, и в которых $\forall i_0, i_1, \; ... \; , i_{n}, j \in M^{n + 2} \; Pr\{\xi_{n + 1} = j | \xi_0 = i_0, \; ... \; , \xi_n = i_n\} = P_{i_n, j}$, где P - матрица переходных вероятностей - Стохастическая (по строкам) матрица (то есть, такая, что сумма всех значений на любой строке равна 1, и все эти значения неотрицательны) размера $|M| \times |M|$, называются Марковскими процессами или цепями (вообще говоря, Марковский процесс не обязательно имеет дискретное время, но поскольку процессы с непрерывным временем мы не рассматриваем, то для нас это синонимы).\\
Пример: напишем матрицу переходных вероятностей для процесса бросания монетки, пока не выпадет 5 орлов (считаем, что если 5 орлов уже выпало, то мы с вероятностью 1 остаемся в этом состоянии).\\
\begin{table}[H]
\caption{Матрица переходных состояний.}
\label{tabular:TransitionStatesMatrix}
\begin{center}
\begin{tabular}{|c|c|c|c|c|c|c|}
\hline
$\cdot$ & 0 & 1 & 2 & 3 & 4 & 5\\
\hline
0 & $\frac{1}{2}$ & $\frac{1}{2}$ & 0 & 0 & 0 & 0\\
\hline
1 & 0 & $\frac{1}{2}$ & $\frac{1}{2}$ & 0 & 0 & 0\\
\hline
2 & 0 & 0 & $\frac{1}{2}$ & $\frac{1}{2}$ & 0 & 0\\
\hline
3 & 0 & 0 & 0 & $\frac{1}{2}$ & $\frac{1}{2}$ & 0\\
\hline
4 & 0 & 0 & 0 & 0 & $\frac{1}{2}$ & $\frac{1}{2}$\\
\hline
5 & 0 & 0 & 0 & 0 & 0 & 1\\
\hline
\end{tabular}
\end{center}
\end{table}
