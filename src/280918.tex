\subsection{28.09.18}
\subsubsection{Лемма 2}
В любом непустом конечном множестве A, частично упорядоченном отношением R существует минимальный элемент. \\
Доказательство: в пункте "Минимальный/наименьший/максимальный/наибольший элемент." предыдущей лекции.
\subsubsection{Теорема о существовании топологической сортировки}
A - произвольное конечное частично упорядоченное отношением R множество. Тогда на множестве A существует топологическая сортировка T, согласованная с R.\\
Доказательство:\\
Обозначим $A_0 = A$.\\
$\forall i \in \mathbb{N}_0, i \leq |A|$ если $A_i = \varnothing$, то его топологическая сортировка $T_i = \varnothing$\\
Если же $A_i$ непустое, то по лемме 2 в нем существует минимальный элемент $m_i$. Тогда определим $A_{i + 1} = A_i \setminus \{m_i\}, T_i = \{(m_i, a) : a \in A_{i}\} \cup T_{i + 1}$. Докажем, что $T_i$ является линейным порядком на $A_i$ и согласовано с R (определение топологической сортировки), а значит, $T = T_0$ является топологической сортировкой $A_0 = A$.\\
Лемма *: если $(a, b) \in R$, то $\exists i, j \in 0:(|A| - 1), \; i \leq j: \; a = m_i, b = m_j$. Доказательство: i и j существуют по построению T, $i \leq j$ так как если $j < i$, то $a \in A_j, a \not= b$, но тогда b не является минимальным элементом в $A_j$,так как $(a, b) \in R$. Стоит также заметить, что поскольку $i \leq j$, то $b \in A_i$, то есть, $(a, b) \in T$ (cогласованность T с R).\\
Рефлексивность: по определению $T_i$.\\
Транзитивность: $\forall a, b, c \in A$, $(a, b) \in T$, $(b, c) \in T$, значит, по построению T $\exists i, j, k \in 0:(|A| - 1)$ такие, что $a = m_i$, $b = m_j$, $c = m_k$, причем $i < j$ и $j < k$, значит, $i < k$, значит, $c \in A_i$, значит, $(a, c) \in T$.\\
Антисимметричность: Пусть $\exists a, b \in A \; : \; (a, b) \in T, (b, a) \in T$, значит, по построению T $\exists i, j \in 0:(|A| - 1)$ такие, что $a = m_i$, $b = m_j$. Так как $(a, b) \in T$, $i \leq j$, а так как $(b, a) \in T$, $j \leq i$, то есть, $i = j$, то есть, $a = b$. \\
Линейный порядок:$\forall a, b \in A \; \exists i, j \in 0:(|A| - 1): a = m_i, b = m_j$. Если $i < j$, то $(a, b) \in T$, иначе $(b, a) \in T$ по построению T. \\
Cогласованность: доказана в лемме *.
\subsubsection{Цепь}
Цепью на множестве A, (строго) частично упорядоченном отношением R, называют всякое $X \subseteq A$, линейно упорядоченное сужением R(X). Если цепь имеет наибольший элемент, говорят что цепь заканчивается в этом элементе.
\subsubsection{Глубина элемента}
Глубиной элемента a множества A, (строго) частично упорядоченного отношением R будем называть максимальную длину цепи, заканчивающейся в a. Глубина a может быть равна бесконечности.
\subsubsection{Расписание}
Расписанием на строго частично упорядоченном отношением R множестве A будем называть такое разбиение $(A_1, A_2, \; ... \;, A_n)$, что $\forall (a, b) \in R, \; b \in A_k \Rightarrow \; \exists j < k$, что $a \in A_j$ 
\subsubsection{Теорема о кратчайшем расписании}
A - конечное, строго частично упорядоченное отношением R множество, h - максимальная глубина элемента в A, $\forall i \in 1:h \; A_i = \{a \in A: depth(a) = i\}$, тогда $\mathbb{A} = (A_1, A_2, \; ... \;, A_n)$ - кратчайшее расписание (расписание наименьшей мощности). Доказательство: \\
$\forall i \in 1:h \; A_i \not= \varnothing$, так как: возьмем цепь $(z_1, \; ... \; , z_h)$ длины h (такая цепь существует, так как h - максимальная глубина элемента в A. $depth(z_h) = h$. $\forall i \in 2:(h-1) \;depth(z_i) = i \Rightarrow depth(z_{i - 1}) = i - 1$. Глубина не может быть меньше i - 1, так как глубина $z_i$ равна i. Если же глубина $z_{i - 1}$ хотя бы i, то существует цепь, не содержащая (по транзитивности и антисимметричности R) $z_i$ длины i, заканчивающаяся в $z_{i - 1}$, но тогда к этой цепи можно дописать $z_i$ и получить цепь длины $i + 1$, заканчивающуюся в $z_i$, что противоречит индукционному предположению о том, что глубина $z_i$ равна i. Значит, $\forall i \in 1:h \; A_i \not= \varnothing$ \\
$\forall i \not= j \; A_i \cap A_j = \varnothing$, так как глубина элемента определяется единственным образом. \\
На данный момент, мы доказали, что $\mathbb{A}$ - разбиение. Теперь нужно доказать, что оно является расписанием. \\
Так как $\mathbb{A}$ - разбиение, $\forall a, b \in A \; \exists i : \; a \in A_i, \; \exists j : \; b \in A_j$ \\
$(a, b) \in R \Rightarrow depth(a) < depth(b) \Rightarrow i < j$ по построению $\mathbb{A}$, то есть $\mathbb{A}$ - расписание.\\
Теперь докажем, что оно кратчайшее. Поскольку $|\mathbb{A}| = h$, то если $\mathbb{A}$ - не кратчайшее расписание, то должно существовать расписание мощности меньше h. Но поскольку h - глубина какого-то элемента в A, существует цепь длины h, а по определению расписания два элемента одной цепи не могут лежать в одном элементе расписания, следовательно, по принципу Дирихле, расписания с мощностью меньше h не существует, следовательно, $\mathbb{A}$ - кратчайшее расписание.
