\subsection{16.11.18}
(Большая часть этой лекции состояла из работы с задачей про хоккейную команду, что я включил в конспект прошлой лекции и из примера про вытаскивание шариков из урн, который мне разбирать лень. Так что всего один пункт. Всегда бы так, а.
\subsubsection{Свойства условной вероятности, формула Байеса}
\begin{itemize}
\item $Pr((A \cap B)|C) = Pr(A|(B \cap C)) * Pr(B|C)$. Доказывается проверкой по формуле условной вероятности.\\
\item Формула Байеса: $Pr(B|A) * Pr(A) = Pr(A \cap B) = Pr(A|B) * Pr(B)$. Доказывается проверкой по формуле условной вероятности.\\
\item $Pr(A_1 \cap \; ... \; \cap A_n) = Pr(A_1|(A_2 \cap \; ... \; \cap A_n)) * \; ... \; * Pr(A_{n - 1}|A_n) * Pr(A_n)$. Доказывается по индукции.\\
\end{itemize}
