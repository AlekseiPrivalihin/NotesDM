\subsection{22.03.19}
\subsubsection{Неорграфы}
$2_V = \{\{u, v\}: \; u, v \in V\}$.\\
Неориентированным графом (неорграфом) называется пара $G = (V, E)$, где $V \not= \varnothing$ - (конечное) множество вершин, а $E \subseteq 2_V$ - множество ребер.\\
Замечание: штрихи, волны, точки и прочая ересь в обозначениях "наследуются"\, то есть мы по умолчанию считаем, что $V'$, например - множество вершин графа $G'$, а множество ребер этого графа - $E'$.\\
Ребра часто записываются по их концам: $(v_1, v_2)$ или просто $v_1v_2$.
Возможна запись $V(G)$ и $E(G)$.
Примечание: \\
хотя $V = \varnothing$ технически незаконно, иногда оно встречается в доказательствах.\\
Пустым графом называют граф, множество ребер которого пусто.\\
Пути, простые пути, циклы, простые циклы, подграфы и порожденные подграфы определяются так же, как для орграфов, с поправкой на замену $V \times V$ на $2_V$.\\
Примечание: цикл $v_1v_2v_1$ в этом определении считается законным, хотя по-моему это контринтуитивно.\\
$u, v \in V$ называются смежными, если $uv \in E$.
\subsubsection{Связность}
Неорграф называется связным, если между любыми двумя его вершинами существует путь.\\
Компонентой связности называется максимальный по включению связный порожденный подграф.\\
По тем же соображениям, что и с компонентами сильной связности, компоненты связности либо совпадают, либо не пересекаются.\\
Утверждение: если $G = (V, E)$ - связный граф, то существует нумерация вершин $v_1, v_2, \; ... \; , v_n$ такая, что $\forall i \in 1:n \; G_i = G[\{v_1, \; ... \; , v_i\}]$ - связный.\\
Доказательство:\\
база индукции - граф из одной вершины связен по определению.\\
шаг индукции - пусть $G_1, \; ... \; , G_i$ построены и связны.\\
Пусть $v \in V \setminus \{v_1, \; ... \; , v_i\}$ (если разность пуста, то $i = n$ и утверждение уже доказано).\\
В силу связности G существует путь $x_0x_1...x_m$, где $x_0 = v, x_m = v_1$.\\
$s = \max\{j \in 0:(m - 1) : \; x_j \not\in \{v_1,\;...\;, v_i\}\}$. Тогда положим $v_{i + 1} = x_s$. $G_{i + 1}$, очевидно, связный.\\
\subsubsection{Деревья}
Связный неорграф без циклов называется деревом.\\
Для следующего утверждения нам нужно научиться удалять и добавлять ребра.\\
Пусть $G = (V, E), \; u, v \in V$. \\
$G - uv = (V, E \setminus \{uv\})$\\
$G + uv = (V, E \cup \{uv\})$\\\\\\\\
Утверждение: следующие условия эквивалентны:\\
\begin{enumerate}
\item T - дерево.\\
\item Любые две вершины T соединяются единственным путем.\\
\item T - минимальный связный граф (в том смысле, что удаление любого ребра нарушит связность).\\
\item T - максимальный граф без циклов (в том смысле, что добавление любого ребра создаст цикл).\\
\end{enumerate}
Доказательство:\\
$1 \Rightarrow 2$) Любые две вершины соединяются хотя бы одним путем, так как дерево связно. Путей не может быть больше одного, так как если есть два различных пути, то из их объединения можно вычленить цикл (это очень нестрого и нужно бы формализовать).\\
$2 \Rightarrow 3$) Если удалить ребро $uv$, то поскольку это был единственный путь между u и v, эти две вершины больше не будут связными.\\
$3 \Rightarrow 4$) В минимальном связном графе нет циклов, так как иначе удаление любого ребра из цикла не нарушает связности. Добавление ребра $uv$ создает цикл, так как между u и v уже был путь.\\
$4 \Rightarrow 1$) Максимальный граф без циклов связен (и, как следствие, является деревом) так как если бы какие-то две вершины u и v нельзя было соединить путем, добавление ребра uv не создавало бы циклов.\\
Произвольный граф без циклов называют лесом (потому что каждая его компонента связности - дерево).\\
\subsubsection{Остовные деревья}
$H \leq G$ называют остовом G, если $V(H) = V(G)$.\\
Утверждение:\\
У каждого графа существует остовный лес (а у связного графа - остовное дерево), не увеличивающее количество компонент связности.\\
Доказательство:\\
Докажем только утверждение про связный граф, так как для несвязного графа можно по отдельности рассмотреть компоненты связности.\\
Для связного графа вспомним чудесную нумерацию из пункта про связность. Будем для каждого $i > 1$ брать ровно одно ребро, связывающее $v_i$ с $G_{i - 1}$. В графе не будет циклов, так как из каждой вершины только одно ребро ведет в вершину с меньшим номером, а если предположить, что цикл таки существует, у вершины с наибольшим номером в цикле должно быть два ребра в вершины с меньшими номерами. Ну и связность построенного графа можно проверить по индукции.\\
Утверждение:\\
(надеюсь, я правильно его понял)\\
Пусть T - дерево, тогда существует такая нумерация вершин $V(T) = \{v_1, \; ... \; , v_n\}$, что $\forall i \in 1:n \; \exists! j \in 1:(i - 1)$ такое, что $v_iv_j \in E(T_i)$, где $T_i = T[\{v_1, \; ... \; , v_i\}]$.\\
Доказательство:\\
Так как T связный, существует такая нумерация, что все $T_i$ связные.\\
В частности, поскольку $T_i$ связный, существует хотя бы одна вершина в $\{v_1, \; ... \; , v_{i - 1}\}$ смежная с $v_i$\\
Если есть две вершины в $\{v_1, \; ... \; , v_{i - 1}\}$ смежные с $v_i$, то поскольку $T_{i - 1}$ связный, это означает наличие цикла в T, чего быть не может.\\
Значит, такая вершина ровно одна.\\
Утверждение: связный граф на n вершинах дерево $\Leftrightarrow$ в нем ровно $n - 1$ ребро.\\
$\Rightarrow$) очевидно следует из нумерации.\\
$\Leftarrow$) Исходный граф $G: |V(G)| = n, |E(G)| = n - 1$. Из следствия, так как граф связный, он имеем остовное дерево $T: |V(T)| = n, |E(T)| = n - 1 \Rightarrow G = T \Rightarrow G$ - дерево\\
\subsubsection{Минимальное остовное дерево}
Введем функцию $w: \; E(G) \rightarrow R_+$ - веса ребер.\\
Вес графа $w_G = \sum\limits_{e \in E}w(e)$\\
Минимальным остовным деревом называется остовное дерево с минимальным весом.\\
Основная лемма об остовных деревьях:\\
Пусть $\mathcal{T}$ - множество всех минимальных остовных деревьев связного графа G.\\
$T \in \mathcal{T}, X \subseteq E(T)$. Пусть $\varnothing \not= S \subseteq V(G)$, $Q = \{uv: \; u \in S, v \in V(G) \setminus S\}$, причем $X \cap Q = \varnothing$.\\
Выберем $e \in Q: \; w(e) = \min\limits_{q \in Q}w(q)$. Тогда $\exists T' \in \mathcal{T}: \; X \cup \{e\} \in E(T')$\\
Доказательство:\\
Если $e \in E(T)$, $T' = T$. Иначе $T + e$ содержит цикл (поскольку дерево - максимальный граф без циклов).\\
Тогда $\exists e' \in E(T): \; e' \in Q$. $e' \not\in X$, так как $X \cap Q = \varnothing$.\\
$T' = T + e - e'$ - остовное дерево (вершины мы не удаляли, связность не нарушалась, так как удалили ребро из цикла.\\
Так как $w(e) = \min\limits_{q \in Q}w(q)$, то $w(e) \leq w(e')$ и $w(T') = w(T) + w(e) - w(e') \leq w(T)$, а значит, поскольку $T$ - минимальное остовное дерево, $T'$ - также минимальное остовное дерево.
