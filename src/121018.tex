\subsection{12.10.18}
\subsubsection{Комбинаторное доказательство и пончики}
Комбинаторное доказательство - способ нахождения мощности множества путем установления биекции с множеством известной мощности. Ничего не понятно? Пончики to the rescue!\\
Пусть у нас есть пончики 5 видов: с ванилью (В), шоколадом (Ш), карамелью (К), сгущенкой (С) и сахарной пудрой (П).\\
Пример 1: пусть мы хотим найти количество способов купить не более чем по одному пончику каждого вида. Тогда установим биекцию между множеством способов купить пончики и множеством двоичных кортежей длины 5. Для каждого кортежа вида $(0, 1, 0, 0, 1)$ скажем, что 0 выглядит как пончик, а потому определим, что если на i-й позиции в кортеже стоит 0, то мы купили пончик такого вида, а если 1 - то нет. То есть, в приведенном выше кортеже мы купим ванильный пончик, карамельный пончик и пончик со сгущенкой. Таким образом, мы построили биекцию (это, если очень хочется, можно по определению биекции проверить. А мне вот не хочется, мне хочется пончиков)между множеством способов купить не более чем по одному пончику каждого вида и множеством двоичных кортежей длины 5, мощность которого, как известно, равна $2^5 = 32$.\\
Пример 2: пусть мы хотим купить 12 пончиков, неважно какого вида. Покажем, что количество способов сделать это равно количеству двоичных кортежей длины 16, в которых ровно 4 единицы. Идея проста: нули в кортеже снова обозначают пончики, причем нули, идущие до i-й единицы отвечают за количество пончиков i-го вида (можно считать, что в конце кортежа стоит еще 17-й элемент, всегда равный единице, но поскольку он зафиксирован, на подсчет количества вариантов он не влияет). Таким образом, $(0, 0, 1, 1, 0, 0, 0, 0, 0, 1, 0, 0, 0, 1, 0, 0)$ соответствует варианту 2 В, 0 Ш, 5 К, 3 С, 2 П. Опять-таки, если вы еще не захлебнулись слюной, можете руками проверить, что это биекция. Ну а количество кортежей длины 16 с ровно 4 единицами равно $C_{16}^4$