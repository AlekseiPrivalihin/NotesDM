\subsection{15.03.19}
Замечание: для понимания этой лекции необходимо знание некоторых определений из следующего раздела.
\subsubsection{Классификация состояний Марковской цепи}
Пусть есть Марковская цепь с множеством состояний S и матрицей переходных вероятностей P, $i, j \in S$.
Говорят, что состояние j достижимо из состояния i ($i \rightarrow j$), если $\exists n \geq 0: \; P^n_{ij} > 0$.\\
Достижимость: \\
\begin{itemize}
\item рефлексивна: $P^0_{ii} = E_{ii} = 1 > 0$\\
\item транзитивна: $i \rightarrow j, j \rightarrow k \Rightarrow \exists m, n \in N_0: \; P^m_{ij} > 0, P^n_{jk} > 0 \Rightarrow P^{n + m}_{ik} = \sum\limits_{l \in S} P^m_{il}P^n_{lk} \geq P^m_{ij}P^n_{jk} > 0$.\\
\end{itemize}
Если $i \rightarrow j$ и $j \rightarrow i$, говорят, что состояния i и j сообщаются ($i \leftrightarrow j$).\\
Отношение сообщения, очевидно, симметрично, а значит, является отношением эквивалентности.\\
Значит, S разбивается на непересекающиеся классы сообщающихся вершин - классы эквивалентности по отношению сообщения.
Очевидно, что при представлении Марковской цепи в виде графа (вершины - состояния, веса ребер - переходные вероятности) классы сообщающихся вершин будут компонентами сильной связности.
Граф, построенный на компонентах сильной связности, ациклический, а значит, существует частичный порядок компонент сильной связности, задаваемый отношением достижимости (если компонента A достижима из компоненты B, то B "меньше" A).\\
Максимальные элементы в таком частичном порядке называются эргодическими классами, а состояния Марковской цепи, попавшие в какой-либо эргодический класс, называются возвратными.\\
Теорема (без доказательства):\\
Пусть $\xi_0 = i \in S$, причем i лежит в каком-то эргодическом классе.\\
$\tau_i$ - первый момент дискретного времени, когда марковский процесс вернется в состояние i, начавшись из i ($\tau_i > 0$). Утверждается, что $Pr\{\tau_i < \infty\} = 1$.\\
Все состояния, не являющиеся возвратными, называют несущественными.\\
Если граф Марковской цепи состоит из единственного эргодического класса, Марковская цепь неприводима (пример про скрещивание с гибридом).\\
Если некоторый эргодический класс состоит из одной вершины, это состояние называют поглощающим (пример ниже).
\subsubsection{Пример со случайным блужданием и поглощающими состояниями}
Пусть множество состояний - натуральные числа от 1 до N. Вероятность перейти в соседнее справа число - p, в соседнее слева - $1 - p$. Однако, состояния 1 и N являются поглощающими, то есть, при попадании в них мы с вероятностью 1 останемся в них же. Или, более формально,\\
$S = 0:N$, $P_{00} = P_{NN} = 1$, $P_{(i - 1)i} = p$, $P_{(i + 1)i} = 1 - p$.
Можно самостоятельно нарисовать граф для этого примера, а также для примера с бросанием монетки до 5 орлов (состояние 5 орлов считаем поглощающим) и примера про скрещивание с гибридом.
\subsubsection{Периодичность класса сообщающихся состояний}
Для состояния $i \in S$ рассмотрим множество $\rho_i = \{n \in N: \; P^n_{ii} > 0\}$.\\
$d(i) = \gcd(\rho_i)$ называют периодом состояния i.\\
В предыдущем примере $d(0) = d(N) = 1, \forall i \in 1:(N - 1) \; d(i) = 2$.\\
Если $d(i) = 1$, состояние i называют непериодическим.\\
Лемма:\\
Периоды сообщающихся состояний равны (а значит, период - свойство класса сообщающихся состояний).\\
Доказательство:\\
Пусть $i, j \in S, i \leftrightarrow j \Rightarrow \exists m, n \in N: \; P^m_{ij} > 0, P^n_{ji} > 0$.\\
$\forall d$ - делитель $\rho_i$, $\forall k \in \rho_j \; P^k_{jj} > 0$.\\
$P^{m + k + n}_{ii} \geq P^m_{ij}P^k_{jj}P^n_{ji} > 0 \Rightarrow (m + k + n) \in \rho_i \Rightarrow (m + k + n) \vdots d$.\\
При этом, $P^{m + n}_{ii} \geq P^m_{ij}P^n_{ji} > 0 \Rightarrow (m + n) \in \rho_i \Rightarrow (m + n) \vdots d$.\\
Если $(m + n) \vdots d$ и $(m + n + k) \vdots d$, значит, $k \vdots d$, то есть любой элемент $\rho_j$ делится на любой делитель $\rho_i$, а значит, $d(j) = \gcd(\rho_j) \geq \gcd(\rho_i) = d(i)$.\\
Аналогично, $d(i) \geq d(j)$, а значит, $d(i) = d(j)$, что и требовалось доказать.\\
Утверждение (без доказательства):\\
Если $d(i) < \infty$, то $\exists N: \; \forall n \geq N \; P^{n * d(i)}_{ii} > 0$.\\
Частный случай: $d(i) = 1 \Leftrightarrow \exists N: \; \forall n > N \; P^n_{ii} > 0$.
