\subsection{19.10.18}
\subsubsection{Мощность множества всех подмножеств}
A - произвольное конечное множество, $|A| = m$.\\
$\phi : \; A \rightarrow 1:m$, если $\phi$ - биекция, то такое $\phi$ называется нумерацией элементов A.\\
Теперь во многих задачах мы можем вместо произвольных множеств мощности m использовать 1:m.\\
Рассмотрим $X = 1:m$, $S \subseteq X$. Определим отображение $\chi : \; 2^X \rightarrow \{0,1\}^m$ из множества всех подмножеств X в множество всех n-арных двоичных кортежей следующим образом: \\
$\forall S \in 2^X \; \chi(S) = (\chi(S)_1, \; ... \; , \chi(S)_m)$, где $\chi(S)_i = [^{1, \; i \in S}_{0, \; i \not\in S}$\\
$\chi$ по построению биекция (по подмножеству единственным образом строится кортеж, по кортежу единственным способом восстанавливается подмножество). Теперь построим отображение $\psi: \; \{0, 1\}^m \rightarrow 0:(2^m - 1)$ таким образом: $\psi(x) = \sum\limits_{i = 1}^{m}{x_i * 2^{m - i}}$, где $x = (x_1, x_2, \; ... \; , x_m)$. $\psi$ тоже биекция (доказательство - перевод числа из десятичной системы в двоичную и обратно).\\
Таким образом, мы построили биекцию между A и 1:m, между $2^{1:m}$ и $\{0, 1\}^m$, и наконец между $\{0, 1\}^m$ и $0:(2^m - 1)$. Тогда $|2^A| = |2^{1:m}| = \{0, 1\}^m = |0:(2^m - 1)| = 2^m$.
\subsubsection{Перестановки}
Перестановкой множества 1:n называется $<a_1, a_2, \; ... \; , a_n>$, где $\forall i \in 1:n \; a_i \in 1:n, \; \forall j \in (i + 1):n \; a_i \not= a_j$\\
Множество всех перестановок множества 1:n обозначается $<1:n>$.
\subsubsection{Подсчет количества перестановок}
Пусть $a \in <1:n>$ - перестановка. Тогда определим ее ключ $T(a) = (t^a_1, \; ... \; , t^a_n)$ следующим образом: $t^a_i = |\{j \in (i + 1):n \; : \; a_j < a_i\}|$, то есть, количество элементов перестановки, стоящих после i-го, которые при этом меньше i-го элемента перестановки. (Например, у перестановки $<3, 1, 2>$ ключ будет $(2, 0, 0)$).\\
Утверждение 1: $a, b \in <1:n>, a \not= b \Rightarrow T(a) \not= T(b)$.\\
Доказательство: так как перестановки различны, $\exists k \in 1:n$ - наименьший индекс, в котором перестановки различаются ($a_k \not= b_k, \forall i \in 1:(k - 1) \; a_i = b_i$). Не умаляя общности, скажем, что $a_k < b_k$ (иначе поменяем перестановки местами).\\
Так как $\{a_1, \; ... \; , a_{k - 1}\} = \{b_1, \; ... \; , b_{k - 1}\}$, $\{a_k, \; ... \; , a_n\} = \{b_k, \; ... \; , b_n\}$.\\
Следовательно, $\forall x \in \{a_{k + 1}, \; ... \; , a_n\}, x < a_k \Rightarrow x \in \{b_{k + 1}, \; ... \; , b_n\}, x < b_k$. Тогда, по построению T, $t^a_k \leq t^b_k$. Но при этом $a_k \in \{b_{k + 1}, \; ... \; , b_n\}, a_k < b_k$, а значит, $t^a_k < t^b_k$, следовательно $T(a) \not= T(b)$.\\
Утверждение 2: $a, b \in <1:n>, T(a)\not= T(b) \Rightarrow a \not= b$.\\
Доказательство: Определим множество $\Pi^n = \{(\pi_1, \; ... \; , \pi_n) \; : \; \forall i \in 1:n \; 0 \leq \pi_i \leq (n - i)\}$. Все ключи перестановок T(a) $\forall a \in <1:n>$ попадают в это множество по построению. Осталось показать, что по ключу можно единственным способом восстановить перестановку.\\
Прежде чем формализовать алгоритм восстановления, рассмотрим его на примере:\\
Пусть $T = (3, 6, 3, 2, 4, 3, 1, 0, 0)$. Первый элемент перестановки должен лежать в множестве $\{1, 2, 3, 4, 5, 6, 7, 8, 9\}$, и в этом множестве должно быть ровно три элемента меньше него. Получается, на первом месте в перестановке должно стоять число 4. Теперь нужно найти в множестве оставшихся чисел $\{1, 2, 3, 5, 6, 7, 8, 9\}$ число такое, что в этом множестве есть ровно шесть чисел меньше его. Таким числом будет 8. Повторяя данный процесс, получим перестановку $<4, 8, 5, 3, 9, 7, 2, 1, 6>$, причем поскольку все элементы перестановки различны, перестановку можно было восстановить единственным способом. Теперь формализуем алгоритм:\\
Дано: $T \in \Pi^n$\\
Задача: восстановить перестановку $a \in <1:n>$ такую, что $T(a) = T$.
Определим множество $p^i$ элементов множества 1:n, которые еще не получили свое место в перестановке ($p^1 = 1:n$). Поскольку на каждом шаге алгоритма в перестановку будет добавляться один элемент, $|p^i| = n - i + 1$. Пронумеруем элементы $p^i$ по возрастанию ($\forall k, j \in 1:(n - i + 1), k < j \Rightarrow p^i_k < p^i_j$). Тогда получается, что элемент множества $p^i$ такой, что в множестве $p^i$ содержится ровно $T_i$ элементов, меньших него, - это $p^i_{T_i + 1}$. Тогда $a_i = p^i_{T_i + 1}$. Множество элементов перестановки, стоящих в перестановке после $a_i$ будет равно $p^{i + 1} = p^i \setminus \{p^i_{T_i + 1}\}$, и в нем будет ровно $T_i$ элементов меньше $a_i$, то есть $T(a)_i = T_i$. Переходим к следующему шагу алгоритма.\\
\subsubsection{Факториальная система счисления}
Забивайте косяки, заваривайте алтайские травы, закидывайтесь грибочками, они вам пригодятся.\\
Итак, кроме привычных систем счисления с постоянным основанием (двоичная, десятичная, шестнадцатиричная и.т.д) существуют системы счисления со смешанным основанием. В данном случае, мы будем рассматривать факториальную систему счисления, то есть такую, что $n_{factorial} = \phi_m\phi_{m - 1}\:...\:\phi_1$, $n_{10} = \sum\limits_{i = 1}^{m}{\varphi_i * i!}$ и $\forall i \in 1:m \; 0 \leq \varphi_i \leq i$.\\
Утверждение: любое натуральное число представимо в факториальной системе счисления единственным образом.\\
Доказательство: Посмотрим на множество $\Pi^n$, которое мы определили в предыдущем пункте. С одной стороны, мы построили биекцию из него в множество $<1:n>$. C другой, очевидным образом строится биекция из $\Pi^n$ в множество чисел, запись которых в факториальной системе счисления имеет меньше чем n знаков (выкинуть из любого $X \in \Pi^n$ ведущие нули и последний ноль - и мы получим некоторое число в факториальной системе счисления, в котором не более чем $n - 1$ знак. Обратно - выписать все цифры числа в факториальной системе счисления, дописать в конце ноль, а потом дописывать ведущие нули, пока не получим длину n). \\
Доказать, что все числа, запись которых в факториальной системе счисления имеет менее n знаков, меньше $n!$, можно по индукции: \\
Для n = 2 максимальным числом,  в записи которого меньше n знаков, будет 1, что меньше $2! = 2$.\\
Если максимальное число $m_n$, в записи которого менее n знаков, меньше $n!$, то при переходе к n + 1 получим: $m_n + n! * n < n! * (n + 1) = n! * n + n!$.\\
Теперь докажем, что все числа, меньшие $n!$, представимы в факториальной системе счисления с менее чем n знаками.\\
Сначала у нас есть число $a < n!$. Обозначим $a_0 = a, n_0 = n$. Теперь найдем максимальное $k_0$ такое, что $k_0 * (n_0 - 1)! \leq a_0$. $k_0 < n_0$, так как иначе $a_0 \leq k_0 * (n_0 - 1)! \geq n_0!$. При этом, $a_1 = a_0 - k_0 * (n_0 - 1)! < (n_0 - 1)!$. Тогда можно перейти к $a_1 = a_0 - k_0 * (n_0 - 1)!, n_1 = n_0 - 1$. Алгоритм конечен, так как после $n - 1$ шага мы получим $a_{n - 1} < n_{n - 1}! = (n - (n - 1))! = 1! = 1$, а поскольку $\forall i \in 0:(n - 1) \; a_i \geq 0$, $a_{n - 1} = 0$, тогда $k_0...k_{n - 1}$ - корректное представление $a$ в факториальной системе счисления.\\
Таким образом, мы построили биекцию между множеством $0:(n! - 1)$ и множеством чисел, запись которых в факториальной системе счисления имеет не более n знаков.\\
Тогда для любого натурального m получаем, что поскольку существует некоторое натуральное n такое, что $n! > m$, m представимо в факториальной системе счисления. 
\subsubsection{Нумерация перестановок и перестановка, следующая за данной}
В прошлом пункте мы построили биекцию из множества $<1:n>$ через множество чисел, имеющих не более n знаков в факториальной системе счисления, в множество $0:(n! - 1)$, то есть пронумеровали перестановки из n элементов. Чтобы понять, как устроена эта нумерация, давайте попробуем по перестановке найти следующую за ней.\\
Пример:\\
$a = <3, 8, 7, 6, 2, 4, 9, 5, 1$\\
$T(a) = (2, 6, 5, 4, 1, 1, 2, 1, 0)$\\
Представление в факториальной системе счисления выглядит так: $26541121_{factorial}$. Прибавляем единицу: $26541121_{factorial} + 1_{factorial} = 26541200$ (первый разряд переполнился, т.к. $1 + 1 = 2 > 1$, второй тоже, так как $2 + 1 = 3 > 2$, третий не переполнился).\\
Теперь по получившемуся числу восстановим перестановку b, следующую за a.\\
$T(b) = (2, 6, 5, 4, 1, 2, 0, 0, 0)$.\\
$b = <3, 8, 7, 6, 2, 5, 1, 4, 9>$.\\
Посмотрим, что изменилось в перестановке: "убывающий хвост" (максимальное количество элементов в конце перестановки, идущих в порядке убывания) перестановки и еще один элемент левее изменят свое положение, так как все разряды, соответствующие "убывающему хвосту", переполнятся, а к следующему разряду прибавится единица. Таким образом, если длина убывающего хвоста была равна l, а $T(a)_{n - l} = k$, то последний $l + 1$ элемент T(b) будет выглядеть так: $(k + 1, 0, \; ... \; , 0)$. C точки зрения перестановки, это значит, что:\\
\begin{itemize}
\item первые $n - l - 1$ элементов перестановки останутся неизменны.\\
\item на $(n - l)$-й позиции в b будет стоять минимальный элемент x множества $\{a_j : \; j \geq n - l\}$, такой что $x > k$.\\
\item оставшиеся l элементов множества $\{a_j : \; j \geq n - l\} \setminus \{x\}$ будут стоять в конце перестановки b в порядке возрастания.\\
\end{itemize}
Пример:
$a = <5, 4, 1, 3, 2>$, n = 5\\
Убывающий хвост a - $(3, 2)$ - имеет длину 2.\\
$(n - 2)$-й элемент a равен 1.\\
Минимальный элемент множества $\{a_j : \; j \geq n - 2\} = \{1, 2, 3\}$, больший 1, равен 2.\\
Таким образом, перестановка, следующая за a, имеет вид $<5, 4, 2, 1, 3>$

