\subsection{15.03.19}
\subsubsection{Орграф}
Ориентированным графом или орграфом называют $G = (V, E)$, где $V \not= \varnothing$ - (конечное) множество вершин, а $E \subseteq V \times V$ - множество ребер.\\
Ребра часто записываются по их концам: $(v_1, v_2)$ или просто $v_1v_2$.\\
Множество вершин графа $G$ обозначают $V(G)$.\\
Множество ребер графа $G$ обозначают $E(G)$.\\
Примечание: \\
Хотя $V = \varnothing$ технически незаконно, иногда оно встречается в доказательствах.\\
Пустым графом называют граф, множество ребер которого пусто.\\
\subsubsection{Подграф}
Пусть $G = (V, E)$. $G' = (V', E')$ называют подграфом $G$ ($G'\leq G$), если $V' \subseteq V, E' \subseteq (V' \times V') \cap E$.\\
Если $E' = (V' \times V') \cap E$, подграф называют порожденным. Порожденный подграф обозначают $G[V']$.\\
\subsubsection{Пути и циклы}
Пусть $G = (V, E)$.\\
Путем называется последовательность вершин $v_0v_1...v_n: \; \forall i \in 0:n \; v_i \in V, \forall i \in 1:n \; (v_{i - 1}, v_i) \in E$.\\
Простым называется путь, в котором все вершины различны. Поскольку непростые пути в задачах встречаются редко, здесь и далее слово "путь" обозначает простой путь, если не указано обратного.\\
Циклом называется последовательность вершин $v_0v_1...v_n: \; \forall i \in 0:n \; v_i \in V, (v_i, v_{i + 1}) \in E, v_0 = v_n$.\\
Простым называется цикл, все вершины которого, кроме последней, различны. Так же как и с путями, слово "цикл" обозначает простой цикл, если не указано обратного.\\
Ациклическим графом называется орграф без циклов.
\subsubsection{Связность}
$Вершины u, v \in V(G)$ связаны между собой в графе G, если существует путь из $u$ в $v$.\\
Cильно связным называется такой граф $G$, что $\forall u, v \in V(G)$ $u$ связана с $v$ и $v$ связана с $u$.\\
Наибольший по включению сильно связный порожденный подграф называется компонентой сильной связности.\\
Отношение сильной связности, очевидно, рефлексивно (путь из одной вершины легален), транзитивно (простая проверка) и симметрично (по построению), а значит, компоненты сильной связности - это классы эквивалентности, то есть, они либо не пересекаются, либо совпадают.\\
Построим новый граф, в котором вершины соответствуют компонентам сильной связности старого и в новом графе из вершины $A$ в вершину $B$ существует ребро, если в исходном графе в компоненте, соответствующей $A$, была вершина $a$, а в компоненте, соответствующей $B$ - вершина $b$ такие, что $ab$ - ребро исходного графа.\\
Заметим, что построенный граф ациклический. (простая проверка от противного).\\
В ациклическом орграфе все компоненты сильной связности состоят из одной вершины.\\
Отношение связности в ациклическом графе, следовательно, рефлексивно, транзитивно и антисимметрично.\\
Значит, оно является отношением частичного порядка, а значит, существует топологическая сортировка.
