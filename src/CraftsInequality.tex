\subsection{Самостоятельное изучение}
\subsubsection{Неравенство Крафта.}

Пусть задан набор длин $l_{1}, \ldots, l_{m}$, не все обязательно различные. Может ли такой набор оказаться набором длин некоторого префиксного кода? 

\textbf{Теорема.} Для того чтобы набор длин $l_{1}, \ldots, l_{m}$ мог быть набором длин кодовых последовательностей некоторого префиксного кода для алфавита из $m$ символов необходимо и достаточно, чтобы $\sum\limits_{i\in 1:m} 2^{-l_{i}} \leqslant 1$.

\textit{Доказательство.} Необходимость. Существует некоторый префиксный код для алфавита мощности $m$, кодовые последовательности которого имеют длины $l_{1}, \ldots, l_{m}$. Проверим выполнение неравенства Крафта. Обратимся к нашей интерпретации множества кодовых последовательностей как набор всех путей на двоичном дереве от корня к листу. Исходный, корневой, узел назовём нулевым уровнем, а дальше -- последовательно прибавляем номер уровня по мере удаления от нулевого. Каждому узлу $v$ на $t$-м уровне поставим в соответствие число $a(v) = 2^{-t}$ (тогда, например, корню соответствует $2^{-0} = 1$). 

Пусть мы нашли узел $v$ на уровне $t$, не являющийся листом, то есть на уровне $t+1$ есть хотя бы один узел, который получился после ветвения из данного -- назевём их $N(v)$ (это множество, состоящее из одного или двух элементов). Тогда, $a(v) \geqslant \sum\limits_{u \in N(v)} a(u)$ --- неравенство, если один узел, равенство, если два узла.

Просуммируем такие неравенства для каждого не листа: $$\sum\limits_{v \text{ не листья}} a(v) \geqslant \sum\limits_{u \text{ не корень}} a(u)$$.

$$2^{0} \geqslant \sum\limits_{u \text{ листья}} a(u).$$

Обратно, пусть выполнено неравенство. Пусть $l_{1}\leqslant \ldots \leqslant l_{m}$.

$n_{j}$ --- число листьев, которые должны оказаться на уровне $j$: $n_{j} = |\{i: l_{i} = j, i\in 1:m\}|$.

Известно, $$\sum\limits_{i\in 1:m} 2^{-l_{i}} \leqslant 1,$$ значит, $$\sum\limits_{j\in 1:l_{m}} 2^{-{j}} n_{j} \leqslant 1.$$

Тогда, для каждого $j\in 1:l_{m}$, $$n_{j} \leqslant 2^{j} - (2^{j-1}n_{1} +\ldots + 2 n_{j-1}).$$

Пусть $m \neq 1$ (случай одной вершины очевиден), рассмотрим $n_{1}$, выделим на первом уровне вершин $n_{1} \leqslant 2$, на втором уровне останется $2(2 - n_{1})$. Известно, что $n_{2} \leqslant 2^{2} - 2n_{1}$, значит, осталось не меньше, чем требуется для второго уровня.

Пусть на $(j-1)$-м уровне было свободно $$2^{j-1} - (2^{j-2} n_{1} +\ldots + 2 n_{j-2}),$$ известно, что $n_{j-1}$ не больше этой величины. Выделим $n_{j-1}$ узлов, останется $2^{j-1} - (2^{j-2} n_{1} +\ldots + 2 n_{j-2}) - n_{j-1}$, значит, на $j$-м уровне будет $2*(\ldots) =2^{j} - (2^{j-1}n_{1} +\ldots + 2 n_{j-1})$.

Таким образом, строится двоичное дерево с нужным количеством листьев на нужных уровнях, по которому восстановим префиксный код.