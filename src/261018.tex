\subsection{26.10.18}
\subsubsection{Подсчет подмножеств размера k}
Пусть A - произвольное конечное множество, $|A| = n$. $S = \{M \subseteq A \; : \; |M| = k\}$. Чему равна мощность S? \\
$A = \{a_1, \; ... \; , a_n\}$\\
$\forall a_i \in <A> \; a = <a_{i_1}, \; ... \; , a_{i_k}, a_{i_{k + 1}}, \; ... \; , a_{i_n}>$\\
$M = \{a_{i_1}, \; ... \; , a_{i_k}\}$. Так как $M \subseteq A$ и $|M| = k$, $M \in S$.\\
$A \setminus M = \{a_{i_{k + 1}}, \; ... \; , a_{i_n}\} = \overline{M}$\\
$<a_{i_1}, \; ... \; , a_{i_k}> \in <M>$\\
$<a_{i_{k + 1}}, \; ... \; , a_{i_n}> \in \overline{M}$\\
Таким образом, $\forall M' \in S$ можно установить биекцию $f: \; <M'> \times <\overline{M'}> \rightarrow <A>$, определенную следующим образом: $\forall b \in <M'> \; forall c \in <\overline{M'}> \; f((b, c)) = <b_1, \; ... \; , b_k, c_1, \; ... \; , c_{n - k}>$. А если без заумных формул, то это значит, что любую перестановку можно разрезать на перестановку некого подмножества размера k (первые k элементов перестановки) и некую перестановку его дополнения (остальные элементы).\\
Таким образом, комбинаторно получаем равенство $|S| * k! * (n - k)! = n!$ (количество перестановок множества A равно количеству способов выбрать подмножество A размера k, затем упорядочить его элементы и затем упорядочить элементы дополнения этого подмножества в множестве A). Отсюда получаем формулу для искомого $|S|$:\\
$|S| = \frac{n!}{k! * (n - k)!}$. Для таких чисел существует специальное обозначение $C^k_n$ (в англоязычной литературе - $(^n_k)$
\subsubsection{Разбиение на подмножества фиксированного размера}
Заметим, что в прошлом пункте, выбрав подмножество размера k, мы автоматически выбрали и его дополнение размера $n - k$, таким образом построив разбиение A на два упорядоченных (не внутри себя, а в том плане, что если k окажется равно n - k, выбранные подмножества нельзя будет поменять местами. Формулировка не очень удачная, но на лекции было так) подмножества фиксированного размера (k и $n - k$). Количество таких разбиений, соответственно, - $C_n^k$. Теперь обобщим эту задачу:\\
Сколькими способами можно разбить множество A мощности n на m упорядоченных (все в том же смысле, внутри подмножеств никакого порядка нет) подмножеств $A_i$, $|A_i| = k_i$, $\sum\limits_{i = 1}^{m}k_i = n$?\\
Действуя по аналогии с предыдущей задачей, разрезаем перестановку длины n на m перестановок соответствуюших длин, получаем биекцию (благодаря тому, что мы умеем нумеровать элементы множеств, я могу заменить в записи подмножество размера $k_i$ на $1:k_i$, что позволяет не вводить 100500 обозначений) $<1:k_i> \times \; ... \; \times <1:k_n> \rightarrow <A>$, получив равенство \\
$P * k_1! * \; ... \; * k_m! = n!$, где P - искомое число разбиений множества A на m упорядоченных (нутыпонел) подмножеств фиксированных размеров. Отсюда $P = \frac{n!}{\prod\limits_{i = 1}^{m} k_i!}$. Такие числа P обозначают как $C_n^{k_1, \; ... \; , k_m}$ или $(^n_{k_1, \; ... \; , k_m})$\\
Пример: давайте посчитаем количество анаграмм (перестановок букв) слова ПАРАЛЛЕЛЕПИПЕД. Каждой анаграмме можно сопоставить набор множеств позиций на которых стоят конкретные буквы, то есть для слова ПАРАЛЛЕЛЕПИПЕД получим для А множество $\{2, 4\}$, для Е - $\{7, 9, 13\}$ и так далее. Нетрудно заметить, что объединением таких множеств для всех букв, содержащихся в слове, будет $1:n$, где n - длина слова, эти множества можно упорядочить (например, упорядочив по алфавиту соответствующие им буквы) и эти множества не пересекаются (на одной позиции в слове не могут стоять две буквы сразу). Таким образом, количество анаграмм слова ПАРАЛЛЕЛЕПИПЕД равно отношению факториала 14 (длина слова) к произведению факториалов количества вхождений буквы в слово для каждой буквы, содержащейся в слове (2 для А, 3 для Е, 1 для И и.т.д.). Получаем $\frac{14!}{2!3!1!1!3!3!1!} = \frac{14!}{2!3!3!3!}$
\subsubsection{Свойства $C_n^k$}
\paragraph{$C_n^k = C_n^{n - k}$}
$\;$ \\ Нетрудно заметить, что выбрав подмножество размера k множества размера n, мы однозначно выбрали и его дополнение - множество размера $n - k$.\\
\paragraph{$C_n^k = C_{n - 1}^{k - 1} + C_{n - 1}^k$}
$\;$ \\ Зафиксируем какой-то элемент a исходного множества. Мы можем либо взять его в наше подмножество размера k, и тогда нам останется выбрать оставшиеся $k - 1$ элементов из $n - 1$ других элементов множества, либо не брать, и тогда нам нужно набрать все k элементов из других $n - 1$ элементов исходного множества, а поскольку множество способов выбрать подмножество размера k, содержащее a, и множество способов выбрать подмножество размера k, не содержащее a, не пересекаются, мощность их объединения равна сумме их мощностей, то есть $C_n^k = C_{n - 1}^{k - 1} + C_{n - 1}^k$\\
\paragraph{$C_{3n}^{n} = \sum\limits_{r = 0}^{n} C_n^r * C_{2n}^{n - r}$}
$\;$ \\ Пусть в некой карточной игре карты делятся на карты существ и карты заклинаний. Пусть у абстрактного Вадима есть в коллекции n заклинаний и 2n существ, и ему нужно составить колоду размера n. Он мог бы просто наугад напихать туда все подряд одним из $C_{3n}^{n}$ способов, не разбирая, где существа, а где заклинания. Но абстрактный Вадим умен, и он сперва хочет определить тактику, зафиксировав количество r заклинаний в колоде. Для конкретного r Вадим может составить колоду $C_n^r * C_{2n}^{n - r}$ способами. Нетрудно заметить, что r может принимать только значения от 0 до n (потому что у Вадима всего n заклинаний), и сумма количества способов составить колоду с r заклинаниями для всех r равна общему количеству способов составить колоду из n карт из 3n карт коллекции, то есть $C_{3n}^{n} = \sum\limits_{r = 0}^{n} C_n^r * C_{2n}^{n - r}$
\subsubsection{Боль, унижение и формула включений и исключений}
Если в какой-то момент приводимые здесь рассуждения становятся непонятными, попробуйте порисовать диаграммы Эйлера-Венна. Если все еще непонятно, напишите мне. Если вы не знаете, что такое диаграмма Эйлера-Венна, добро пожаловать в Википедию.
Попробуем найти мощность объединения двух множеств. Если эти множества дизъюнктны (то есть, их пересечение пусто), то мощность объединения равна сумме мощностей. Если же пересечение непусто, то лежащие в нем элементы мы посчитали дважды, соответственно, из суммы мощностей надо вычесть мощность пересечения. $|A \cup B| = |A| + |B| - |A \cap B|$.\\
Для трех множеств будем рассуждать аналогично. $|A| + |B| + |C|$ учитывает дважды элементы, лежащие в попарных пересечениях. Но если вычесть мощности попарных пересечений ($|A| + |B| + |C| - |A \cap B| - |A \cap C| - |B \cap C|$), получится, что пересечение всех трех множеств мы учли три раза и вычли три раза (поскольку оно содержится как в каждом из множеств, так и в каждом из попарных пересечений), и нужно добавить его мощность к ответу: $|A \cup B \cup C| = |A| + |B| + |C| - |A \cap B| - |A \cap C| - |B \cap C| + |A \cap B \cap C|$. \\
Попробуем обобщить: мощность объединения n множеств $|A_1 \cup \; ... \; \cup A_n|$ - это сумма, в которую со знаком $+$ входят все мощности пересечения нечетного количества множеств и со знаком $-$ - все мощности пересечения четного количества множеств, то есть \\
$|A_1 \cup \; ... \; \cup A_m| = \sum\limits_{1 \leq j \leq n} (-1)^{j + 1} \sum \limits_{1 \leq i_1 \leq \; ... \; \leq i_j \leq n} |A_{i_1} \cap \; ... \; \cap A_{i_j}|$\\
Доказывать будем по индукции. База индукции: для n = 1 и n = 2 формулу мы уже рассмотрели.\\
Переход: Предположим, что для $n = m$ формула уже доказана, докажем ее для $n = m + 1$. воспользуемся ассоциативностью объединения\\
$|A_1 \cup \; ... \; \cup A_m \cup A_{m + 1}| = |(A_1 \cup \; ... \; \cup A_m) \cup A_{m + 1}|$\\
Теперь воспользуемся формулой для n = 2:\\
$|(A_1 \cup \; ... \; \cup A_m) \cup A_{m + 1}| = |(A_1 \cup \; ... \; \cup A_m)| + |A_{m + 1}| - |(A_1 \cup \; ... \; \cup A_m) \cap A_{m + 1}|$\\
Теперь воспользуемся дистрибутивностью пересечения:\\
$|(A_1 \cup \; ... \; \cup A_m) \cup A_{m + 1}| = |(A_1 \cup \; ... \; \cup A_m)| + |A_{m + 1}| - |(A_1 \cap A_{m + 1}) \cup \; ... \; \cup (A_m \cap A_{m + 1})|$\\
Теперь можно воспользоваться формулой для n = m:
$|(A_1 \cup \; ... \; \cup A_m) \cup A_{m + 1}| = |(A_1 \cup \; ... \; \cup A_m)| + |A_{m + 1}| - \sum\limits_{1 \leq j \leq n} (-1)^{j + 1} \sum \limits_{1 \leq i_1 \leq \; ... \; \leq i_j \leq m} |(A_{i_1} \cap A_{m + 1}) \cap \; ... \; (\cap A_{i_j} \cap A_{m + 1})|$\\
Заметим, что нам достаточно всего один раз пересечь с $A_{m + 1}$, а также внесем минус под скобки:\\
$|(A_1 \cup \; ... \; \cup A_m) \cup A_{m + 1}| = |(A_1 \cup \; ... \; \cup A_m)| + |A_{m + 1}| + \sum\limits_{1 \leq j \leq n} (-1)^{j + 2} \sum \limits_{1 \leq i_1 \leq \; ... \; \leq i_j \leq m} |A_{i_1} \cap \; ... \; \cap A_{i_j} \cap A_{m + 1}|$\\
Теперь посмотрим на эту формулу внимательно: все мощности пересечений, не содержащих $A_{m + 1}$ входят в первое слагаемое со знаком $(-1)^{j + 1}$, где j - количество множеств в пересечении. $|A_{m + 1}|$ - второе слагаемое - имеет знак $(-1)^0 = (-1)^2 = (-1)^{1 + 1}$, то есть $-1$ в степени "количество множеств в пересечении плюс один". Наконец, все пересечения, содержащие более одного множества и содержащие $A_{m + 1}$ входят в последнее слагаемое со знаком $(-1)^{j + 2}$, где j - количество множеств в пересечении, не считая $A_{m + 1}$, то есть со знаком $(-1)^{j' + 1}$, где $j'$ - количество множеств в пересечении. Таким образом, получается, что \\
$|A_1 \cup \; ... \; \cup A_m| = |(A_1 \cup \; ... \; \cup A_m)| + |A_{m + 1}| + \sum\limits_{1 \leq j \leq n} (-1)^{j + 2} \sum \limits_{1 \leq i_1 \leq \; ... \; \leq i_j \leq m} |A_{i_1} \cap \; ... \; \cap A_{i_j} \cap A_{m + 1}| = \sum\limits_{1 \leq j \leq n} (-1)^{j + 1} \sum \limits_{1 \leq i_1 \leq \; ... \; \leq i_j \leq n} |A_{i_1} \cap \; ... \; \cap A_{i_j}|$\\
Формула включений и исключений доказана.