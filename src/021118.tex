\subsection{02.11.18}
\subsubsection{Подсчет количества разбиений, числа Белла и числа Стирлинга 2-го рода}
Пусть есть произвольное конечное множество A, $|A| = n$, нужно подсчитать количество его разбиений мощности k (то есть, таких наборов множеств $X_1, \; ... \; , X_k$, что $\forall i \in 1:k \; X_i \not= \varnothing, \forall j \in 1:k, i \not= j \Rightarrow X_i \cap X_j = \varnothing$).\\
Для $k = 2$ все довольно просто: нужно выбрать элементы, которые пойдут в первое множество разбиения, а все остальные, естественно, пойдут во второе. Таким образом, количество разбиений мощности 2 равно количеству способов выбрать непустое собственное (то есть, не совпадающее со всем множеством) подмножество A пополам (так как мы не упорядочиваем элементы разбиения, а значит, случаи $\{X_1, X_2\}$ и $\{X_2, X_1\}$ мы различать не должны), то есть таких способов $\frac{|2^A| - 2}{2} = \frac{2^n - 2}{2} = 2^{n - 1} - 1$\\
Теперь перейдем к общему случаю:\\
Обозначим $B(n)$ число разбиений множества мощности n, $S(n, k)$ - число разбиений мощности k множества мощности n.\\
Нетрудно понять, что $B(n) = \sum\limits_{i = 1}^{k}S(n, k)$.\\
$B(n)$ называют n-ным числом Белла, а $S(n, k)$ - числом Стирлинга второго рода из n по k. \\
Примечание: поскольку числа Стирлинга второго рода встречаются намного чаще чем числа Стирлинга первого рода, я буду называть их просто числами Стирлинга. Алсо, в англоязычной литературе эти числа обозначаются $\{^n_k\}$\\
Въетнамский флэшбек: как вы помните, у нас уже были числа $C^k_n$, и для них была рекуррентная формула $C^k_n = C^k_{n - 1} + C^{k - 1}_{n - 1}, C^0_n = 1$. Теперь попробуем запилить похожую формулу для чисел Стирлинга.\\
Утверждается, что эта формула имеет вид $S(n, k) = S(n - 1, k - 1) + k * S(n - 1, k)$\\
Доказательство:\\
Произвольное множество A мощности n $A = \{a_1, a_2, \; ... \; , a_n\}$\\
Для каждого разбиения этого множества верно одно из двух:\\
\begin{itemize}
\item либо $\{a_n\}$ - отдельный элемент разбиения, таких разбиений столько же, сколько разбиений мощности $k - 1$ множества мощности $n - 1$, то есть $S(n - 1, k - 1)$\\
\item либо $\exists i \in 1:k \; a_n \in X_i, |X_i| > 1$. Но тогда можно сначала найти разбиение мощности k множества $A \setminus \{a_n\}$ мощности $n - 1$ (способов это сделать $S(n - 1, k)$), а потом добавить $a_n$ в один из элементов разбиения (k способов), то есть всего разбиений, в которых $\exists i \in 1:k \; a_n \in X_i, |X_i| > 1$, $k * S(n - 1, k)$\\
\end{itemize}
Поскольку очевидно, что множество разбиений, подпадающих под первый случай, и множество разбиений, подпадающих под второй случай, не пересекаются, то мощность их объединения равна сумме их мощностей, то есть\\
$S(n, k) = S(n - 1, k - 1) + k * S(n - 1, k)$\\
Свойства чисел Стирлинга:\\
Во-первых, несколько крайних значений, на которые будет опираться наша формула:\\
$S(n, 0) = 0$\\
$S(0, 0) = 0$\\
$S(k, n) = 0$ при $k > n$\\
Ну и пара значений чисто чтобы было:\\
$S(n, 2) = 2^{n - 1} - 1$\\
$S(n, n - 1) = C^2_n$\\
Алсо, если кому не лень, можно порисовать треугольник Паскаля для чисел Стирлинга. Мне лень.
\subsubsection{Рекуррентная формула для чисел Белла}
$B(n + 1) = \sum\limits_{k = 0}^{n} C_n^k * B(k)$\\
Доказательство:\\
Пусть множество $A = \{a_1, \; ... \; , a_n, a_{n + 1}\}$, $A = X_1 \cup \; ... \; \cup X_k$ - произвольное разбиение\\
$\exists i \in 1:k \; a_{n + 1} \in X_i$. $1 \leq |X_i| = j \leq (n + 1)$\\
$|A \setminus X_i| = n + 1 - j$\\
Количество способов набрать $X_i$ равно $C_n^{j - 1} = C_n^{n - (j - 1)} = C_n^{n + 1 - j}$, количество разбиений $A \setminus X_i$ - $B(n + 1 - j)$.\\
То есть, $B(n + 1) = \sum\limits_{j = 1}^{n + 1}C_n^{n + 1 - j} * B(n + 1 - j)$.\\
Ну и чтобы получить исходную формулу, осталось только изменить нумерацию, приняв $k = n + 1 - j$:\\
$B(n + 1) = \sum\limits_{k = 0}^{n} C_n^k * B(k)$
\subsubsection{Явная формула для чисел Стирлинга}
Единственное, что меня беспокоило — это явные формулы. В мире нет никого более беспомощного, безответственного и безнравственного, чем человек, пытающийся вывести явную формулу. И я знал, что довольно скоро мы в это окунёмся. (С)\\
$\forall n \geq 0, k \geq 1 \; S(n, k) = \frac{1}{k!}\sum\limits_{j = 0}^{k}(-1)^j * C_k^j * (k - j)^n$\\
Доказательство:\\
$S(0, k) = 0$. Для $n = 0$ формула имеет вид:\\
$S(n, k) = \frac{1}{k!}\sum\limits_{j = 0}^{k}(-1)^j * C_k^j * (k - j)^0 =$\\
$ = \frac{1}{k!}\sum\limits_{j = 0}^{k}(-1)^j * C_k^j = $\\
$ = \frac{1}{k!}\sum\limits_{j = 0}^{k}(-1)^j * C_k^j * 1^{k - j} = $(в этот момент мы видим формулу бинома Ньютона)\\
$ = \frac{1}{k!}(1 + (-1))^k = 0$\\
А теперь поехали обобщать. $L = \{R \subseteq A \times 1:k \; : \; R$ - сюръекция$\}$ - множество сюръективных отображений из A в 1:k. Нетрудно понять, что это множество равномощно множеству упорядоченных (то есть, где $X_i$ и $X_j$ нельзя менять местами) разбиений мощности k множества A (если $R(a) = i$, то в разбиении $a \in X_i$ и наоборот).\\
Чтобы получить наши родные неупорядоченные разбиения достаточно поделить на $k!$, то есть $S(n, k) = \frac{1}{k!} * |L|$. \\
Теперь найдем мощность L. Для этого найдем мощность множества всех отображений из A в 1:k, а потом вычтем мощность множества всех несюръективных отображений из A в 1:k.\\
Обозначим множество всех отображений из A в 1:k как M. $|M| = k^n$\\
Теперь обозначим для $\forall i \in 1:k$ $P_i = \{R \subseteq A \times 1:k \; : \; R$ - отображение$, \not\exists a \in A \; R(a) = i\}$\\
Тогда $|L| = |M| - |R_1 \cup \; ... \; \cup R_k| = k^n - |R_1 \cup \; ... \; \cup R_k|$\\
$|R_1 \cup \; ... \; \cup R_k|$ найдем по формуле включений и исключений:\\
$|R_1 \cup \; ... \; \cup R_k| = \sum_{j = 1}^{k}(-1)^{j + 1}\sum\limits_{1 \leq i_1 \leq \; ... \; \leq i_j \leq k}|P_{i_1} \cap \; ... \; \cap P_{i_j}|$\\
$|P_{i_1} \cap \; ... \; \cap P_{i_j}| = (k - j)^n$ - количество отображений из A в $1:k \setminus \{i_1, \; ... \; , i_j\}$ - не зависит от выбора конкретных $\{i_1, \; ... \; , i_j\}$, а значит, $\sum\limits_{1 \leq i_1 \leq \; ... \; \leq i_j \leq k}|P_{i_1} \cap \; ... \; \cap P_{i_j}| = C_k^j * (k - j)^n$, поскольку выбрать множество $\{i_1, \; ... \; , i_j\}$ из $1:k$ можно $C_k^j$ способами. Тогда получаем:\\
$|R_1 \cup \; ... \; \cup R_k| = \sum_{j = 1}^{k}(-1)^{j + 1}\sum\limits_{1 \leq i_1 \leq \; ... \; \leq i_j \leq k}|P_{i_1} \cap \; ... \; \cap P_{i_j}| = \sum\limits_{j = 1}^{k} (-1)^{j + 1} * C_k^j * (k - j)^n$\\
Откуда\\
$|L| = k^n - \sum\limits_{j = 1}^{k} (-1)^{j + 1} * C_k^j * (k - j)^n = k^n + \sum\limits_{j = 1}^{k} (-1)^j * C_k^j * (k - j)^n = (-1)^0 * C_k^0 * (k - 0)^n + \sum\limits_{j = 1}^{k} (-1)^j * C_k^j * (k - j)^n = \sum\limits_{j = 0}^{k} (-1)^j * C_k^j * (k - j)^n$\\
Явная формула для чисел Стирлинга доказана.