\subsection{21.09.18}
\subsubsection{Свойства бинарных отношений}
Бинарное отношение R над множеством A называется рефлексивным, если $\forall a \in A \; (a, a) \in R$\\
Бинарное отношение R над множеством A называется антирефлексивным, если $\forall a \in A \; (a, a) \not\in R$\\
Бинарное отношение R над множеством A называется транзитивным, если $\forall a, b, c \in A \; (a, b) \in R, \; (b, c) \in R \Rightarrow (a, c) \in R$\\
Бинарное отношение R над множеством A называется антитранзитивным, если $\forall a, b, c \in A \; (a, b) \in R, \; (b, c) \in R \Rightarrow (a, c) \not\in R$\\
Бинарное отношение R над множеством A называется симметричным, если $\forall a, b \in A \; (a, b) \in R \Rightarrow (b, a) \in R$\\
Бинарное отношение R над множеством A называется антисимметричным, если $\forall a, b \in A \; (a, b) \in R, \; (b, a) \in R \Rightarrow a = b$\\
Бинарное отношение R над множеством A называется асимметричным, если $\forall a, b \in A \; (a, b) \in R \Rightarrow (b, a) \not\in R$\\
\subsubsection{Отношение порядка}
Бинарное отношение R над множеством A называется частичным порядком на A (или говорят, что A частично упорядочено R), если R рефлексивно, транзитивно и антисимметрично.\\
Бинарное отношение R над множеством A называется строгим частичным порядком на A, если R антирефлексивно, транзитивно и асимметрично.\\
Бинарное отношение R над множеством A называется линейным порядком на A, если R является частичным порядком на A и $\forall a, b \in A \; (a, b) \in R \; || \; (b, a) \in R$\\
Бинарное отношение R над множеством A называется строгим линейным порядком на A, если R является строгим частичным порядком на A и $\forall a, b \in A \; (a, b) \in R \; || \; (b, a) \in R$\\
\subsubsection{Эквивалентность частичного порядка и отношения вложения}
Произвольное множество A частично упорядочено R. Тогда существует биекция $f: \; A \rightarrow S$, где $S \subseteq 2^A$, $S = \{\{x \in A : \; (x, a) \in R\} : \; a \in A\}$. \\ 
Доказательство: \\
Определим $f(a) = \{x \in A : \; (x, a) \in R\}$\\
Cюръекция: по определению S.\\
Инъекция: предположим обратное. Тогда $\exists a, b \in A \; a \not= b, \; f(a) = f(b)$. $a \in f(a)$ и $b \in f(b)$ так как R рефлексивно. Значит, также $b \in f(a)$ и $a \in f(b)$. Но тогда по определению f $(a, b) \in R$ и $(b, a) \in R$, и при этом $a \not= b$, что противоречит антисимметричности R.\\
Таким образом, отношение частичного порядка R на множестве A эквивалентно отношению $\subseteq$ вложенности на множестве $S = \{\{x \in A : \; (x, a) \in R\} : \; a \in A\}$, так как $\forall a, b \in A \; (a, b) \in R \Rightarrow f(a) \subseteq f(b)$ и наоборот.
\subsubsection{Минимальный/наименьший/максимальный/наибольший элемент}
Произвольное множество A частично упорядочено R.\\
Минимальным элементом в A называют $m \in A$ такой, что $\not\exists a \in A \; (a, m) \in R$. Минимальный элемент не обязательно единственный. В непустом конечном множестве существует хотя бы один минимальный элемент. (доказательство ниже).\\
Наименьшим элементом в A называют $m \in A$ такой, что $\forall a \in A \; (m, a) \in R$. Наименьший элемент не обязательно существует. Наименьший элемент единственный по предыдущему утверждению.\\
Максимальным элементом в A называют $m \in A$ такой, что $\not\exists a \in A \; (m, a) \in R$. Минимальный элемент не обязательно единственный. В непустом конечном множестве существует хотя бы один максимальный элемент. (доказательство ниже).\\
Наибольшим элементом в A называют $m \in A$ такой, что $\forall a \in A \; (a, m) \in R$. Наибольший элемент не обязательно существует. Наибольший элемент единственный по предыдущему утверждению.\\
Доказательство существования минимального элемента в произвольном непустом конечном множестве A, частично упорядоченном отношением R: \\
Предположим обратное. Тогда $\forall a \in A \; \exists b \in A \; (b, a) \in R$. Тогда (поскольку A непусто) выберем любой элемент x из A и начнем строить цепочку $(x_1, x_2, \; ...)$ такую, что $x_1 = x$, а $\forall i \in N \; i > 1, \; x_i \in A \; (x_i, x_{i - 1}) \in R$. По предположению эта цепочка имеет бесконечную длину и по транзитивности и антисимметричности R элементы в ней не повторяются, но поскольку A - конечное множество, а все элементы цепочки лежат в A, мы получаем противоречие.\\
Доказательство существования максимального элемента в произвольном непустом конечном множестве A, частично упорядоченном отношением R, аналогично.
\subsubsection{Топологическая сортировка}
Топологической сортировкой множества A, частично упорядоченного отношением R, называется такой линейный порядок T на A, что $(a, b) \in R \Rightarrow (a, b) \in T$.\\
\subsubsection{1}
$X \subseteq A \Rightarrow R(X)$ (сужение R на X) сохраняет все свойства R ((анти)рефлексивность, (анти)транзитивность, (а/анти)симметричность). Важно: эти свойства могут появиться при сужении, но не могут исчезнуть (то есть, например, сужение нерефлексивного отношения может быть рефлексивным, а вот сужение рефлексивного нерефлексивным - нет). Доказательство: в лоб проверить по определениям. Мне лень. \\