\subsection{01.09.18}
\paragraph{Множество}
- это коллекция объектов произвольной природы. Обозначаются прописными латинскими либо греческими буквами.
\paragraph{Элементы множества}
- это объекты, составляющие множество.
\subparagraph*{Примеры:}
$\Phi = \{1, \lambda, element, \{IV, white\}\}$ - множество, состоящее из элементов $1, \lambda, element, \{IV, white\}$, причем последний элемент сам является множеством. \\
$1 \in \Phi$ - 1 содержится в $\Phi$ или 1 - элемент $\Phi$\\
$IV \not\in \Phi$ - IV не содержится в $\Phi$, IV не является элементом $\Phi$
\paragraph{Подмножество}
$\;$ \\ A - произвольное множество\\
$B \subseteq A \Leftrightarrow \forall b \in B \; b \in A$, \\
то есть B называют подмножеством A тогда и только тогда, когда любой элемент множества B является также элементом множества A.
\subsubsection{Собственное подмножество}
$\;$ \\ $B \subseteq A$ и $\exists x \in A \; x \not\in B \Leftrightarrow B \subset A$, \\
то есть собственным подмножеством множества A называют такое подмножество B, что в A существует элемент, который не является элементом B.\\
$B \subsetneqq A$ - такая запись может использоваться чтобы подчеркнуть, что B - собственное подмножество A.
\subsubsection{Равенство множеств}
$\;$ \\ Если $B \subseteq A$ и $A \subseteq B$, то $A = B$.
\subsubsection{Мощность множества}
- это число элементов в нем (для конечных множеств). Обозначается как $|A|$
\subsubsection{Пустое множество} - множество, мощность которого равна 0. Обозначается как $\varnothing$
\subsubsection{Способы задания множества}
$\;$ \\
\begin{itemize}
\item Полное перечисление элементов, например $\{1, 2, 3, 4, 5\}$
\item Интуиция. Запись $\{1, 2, \; ... \;, 10\}$ очевидно задает множество натуральных чисел от 1 до 10, хотя формально такая запись смысла не имеет. \\ $\{1, 2, \; ...\}$ задает множество всех натуральных чисел (обозначается как $\mathbb{N}$. Преподаватель удваивает не поперечный штрих, а левый вертикальный, но делать так в Latex я не умею). \\ Запись $n \in \mathbb{N} \; \{1, 2, \; ... \; , n\}$ формально некорректна для $n = 1$, но интуитивно понятно, что в таком случае двойку надо выбросить. Множество натуральных чисел от m до n мы будем обозначать как $m:n$
\item Условие выбора. $\{x \in \mathbb{N} : x \vdots 2\}$ задает множество всех натуральных чисел, которые делятся на 2, то есть $\{2, 4, \; ... \}$. Слева от двоеточия задается множество, откуда мы выбираем элементы, справа - условие выбора. Запись $\{x: <something>\}$ означает, что в множестве содержатся ЛЮБЫЕ объекты, которые удовлетворяют условию, записанному справа от двоеточия.
\item Множества могут быть заданы как результат некоторых операций над множествами.
\end{itemize}
\subsubsection{Операции над множествами}
$\;$ \\ A, B - произвольные множества. $||$ здесь и далее обозначает логическое ИЛИ, $\&\&$ - логическое И.\\
\begin{itemize}
\item Объединение множеств: $A \cup B = \{x : x \in A \;||\; x \in B\}$
\item Пересечение множеств: $A \cap B = \{x : x \in A \;\&\&\; x \in B\}$
\item Разность множеств: $A \setminus B = \{x : x \in A \;\&\&\; x \not\in B\}$
\end{itemize}
\paragraph{Свойства операций над множествами}
$\;$ \\ Объединение и пересечение обладают ассоциативностью и коммутативностью. Доказать это можно проверив соответствующие равенства (например, для ассоциативности объединения - $A \cup (B \cup C) = (A \cup B) \cup C$) по определению равенства множеств.
\paragraph{Запись}
$\;$ \\ $n \in \mathbb{N} \; A_1 \cup A_2 \cup ... \cup A_n = \bigcup\limits_{i \in 1:n} A_i$\\
$n \in \mathbb{N} \; A_1 \cap A_2 \cap ... \cap A_n = \bigcap\limits_{i \in 1:n} A_i$ \\
\subsubsection{Множество всех подмножеств данного множества}
$\;$ \\ Пусть A - произвольное множество. Тогда $\{B : B \subseteq A\} = 2^{A}$\\
$|2^{A}| = 2^{|A|}$ (доказательство этого факта интуитивно понятно, если знать, что такое битовые маски)
\subsubsection{Примеры:}
$\;$ \\ $\{element\} \subseteq 2^{\Phi}$ - ложно, так как $element \not\subseteq \Phi$\\
$\{element\} \in 2^{\Phi}$ - верно, так как $\{element\} \subseteq \Phi$ \\
$\varnothing \in 2^{\Phi}$, так как $\varnothing$ является подмножеством любого множества. \\
$\Phi \in 2^{\Phi}$
\subsubsection{Разбиение множества}
$\;$ \\ A - произвольное множество. $\Lambda = \{\Lambda_1, \Lambda_2, \; ... \; , \Lambda_n\}$, где $n \in \mathbb{N}$ называется разбиением множества A, если:\\
\begin{itemize}
\item $\Lambda \subseteq 2^{A}$
\item $\forall i \in 1 : n \; \Lambda_{i} \not= \varnothing$
\item $\forall i, j \in 1 : n \; i \not= j \; \Lambda_{i} \cap \Lambda_{j} = \varnothing$
\item $\bigcup\limits_{i \in 1 : n}\Lambda_{i} = A$
\end{itemize}
\subsubsection{Измельчение разбиения}
$\;$ \\ A - произвольное множество, $\Lambda$ и K - разбиения A, $|\Lambda| = m$, $|K| = n$. $\Lambda$ называют измельчением K (или говорят, что $\Lambda$ мельче K), если $\forall i \in 1 : m \; \exists j \in 1 : n \; \Lambda_{i} \subseteq K_{j}$ \\
Стоит помнить, что $\Lambda$ всегда мельче $\Lambda$
\subsubsection{Произведение разбиений}
$\;$ \\  A - произвольное множество, $\Lambda$ и K - разбиения A, $|\Lambda| = m$, $|K| = n$. Произведением разбиений K и $\Lambda$ называется такое разбиение $\Pi$ множества A, которое мельче K и мельче $\Lambda$, и при этом самое крупное из этих измельчений (то есть, все разбиения, которые мельче K и мельче $\Lambda$ будут также мельче $\Pi$).
\paragraph{Доказательство существования: }
$\;$ \\
Определим $\Pi_{ij}$ как $\Lambda_{i} \cap K_{j}$, $\Pi_{0}$ как множество всех $\Pi_{ij}$ для $ i \in 1 : m, j \in 1 : n$\\ и докажем, что $\Pi$ является произведением $\Lambda$ и K.\\
Докажем, что $\Pi = \{\Pi_{0} : \Pi_{ij} \not= \varnothing\}$ является произведением $\Lambda$ и K
\begin{itemize}
\item Докажем, что $\forall i \in 1 : m, \; \forall j \in 1 : n \; \Pi_{ij} \subseteq A$. Так как K и $\Lambda$ - разбиения множества A, $\Lambda_{i} \subseteq A, K_{j} \subseteq A \Rightarrow \Pi_{ij} = \Lambda_{i} \cap K_{j} \subseteq A$.
\item Докажем, что $\forall i, p \in 1 : m, \; \forall j, q \in 1 : n, \; i \not= p \; \Pi_{ij} \cap \Pi_{pq} = \varnothing$. $\forall x \in \Pi_{ij} \; x \in \Lambda_{i}$ по определению $\Pi_{ij}$. Так как $\Lambda$ - разбиение множества A, $\Lambda_{i} \cap \Lambda_{p} = \varnothing \Rightarrow x \not\in \Lambda_{p} \Rightarrow x \not\in \Pi_{pq} \Rightarrow \Pi_{ij} \cap \Pi_{pq} = \varnothing$
\item Аналогично докажем, что $\forall i, p \in 1 : m, \; \forall j, q \in 1 : n, \; j \not= q \; \Pi_{ij} \cap \Pi_{pq} = \varnothing$. Таким образом, мы доказали, что пересечение любых двух элементов $\Pi$ - пустое множество. 
\item Докажем, что $\bigcup\limits_{i \in 1 : m, j \in 1 : n}\Pi_{ij} = A$. Во-первых, поскольку, как было доказано выше, $\forall i \in 1 : m, \; \forall j \in 1 : n \; \Pi_{ij} \subseteq A$, $\bigcup\limits_{i \in 1 : m, j \in 1 : n}\Pi_{ij} \subseteq A$. Во-вторых, поскольку K и $\Lambda$ - разбиения множества A, $\forall a \in A \exists i \in 1 ; m, j \in 1 : n \; a \in \Lambda_{i}, a \in K_{j} \Rightarrow a \in \Pi_{ij} \Rightarrow a \in \bigcup\limits_{i \in 1 : m, j \in 1 : n}\Pi_{ij} \Rightarrow A \subseteq \bigcup\limits_{i \in 1 : m, j \in 1 : n}\Pi_{ij}$, то есть по определению равенства множеств, $\bigcup\limits_{i \in 1 : m, j \in 1 : n}\Pi_{ij} = A$. Очевидно, что добавление или изъятие из списка объединяемых множеств любого количества пустых множеств никак не влияет на результат объединения.
\item $\Pi$ мельче $\Lambda$ и мельче K по определению $\Pi$, так как любой элемент $\Pi$ является подмножеством какого-то элемента $\Lambda$ и подмножеством какого-то элемента K.
\item Докажем, что если разбиение $\Omega$ множества A мельче $\Lambda$ и мельче K, то оно мельче $\Pi$. Так как $\Omega$ мельче $\Lambda$ и мельче K, то $\forall \omega \in \Omega \; \exists i \in 1 : m \; \omega \subseteq \Lambda_{i}; \; \exists j \in 1 : n \; \omega \subseteq K_{j} \Rightarrow \omega \subseteq \Pi_{ij}$. $\omega \not= \varnothing$ так как $\Omega$ - разбиение. Следовательно, $\Pi_{ij} \not= \varnothing \Rightarrow \Pi_{ij} \in \Pi$. Таким образом, мы доказали, что произведение произвольных разбиений $\Lambda$ и K произвольного множества A существует.
\end{itemize}