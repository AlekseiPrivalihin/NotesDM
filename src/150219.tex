\subsection{15.02.19}
Примечание: здесь и далее (в том числе, в последующих лекциях) при введении условной вероятности мы подразумеваем, что нужные вероятности положительны, поэтому это условие будет опускаться.
\subsubsection{Конечная случайная схема и энтропия}
Пусть $A_1, A_2, \; ... \; , A_n$ - разбиение множества исходов S вероятностного пространства (S, Pr). \\
Конечной случайной схемой (КСС) называется схема $\alpha$, сопоставляющая каждому $A_i$ вероятность $Pr(A_i)$.\\
Энтропией КСС называется $H(\alpha) = -\sum\limits_{i = 1}^n Pr(A_i) * \log Pr(A_i)$.\\
Некоторые свойства энтропии:\\
\begin{itemize}
\item $H(\alpha) \geq 0$\\
\item Энтропия характеризует неопределенность, заключенную в КСС\\
\item Для любой КСС $\alpha$, у которой k исходов, $H(\alpha) \leq \log k$\\
Доказательство:\\
Обозначим $f(x) = -x * \log x$. На отрезке $[0, 1]$ $f(x)$ строго вогнутая, а значит, по неравенству Йенсена,\\
$\sum\limits_{i = 1}^n \lambda_i * f(x_i) \leq f(\sum\limits_{i = 1}^n \lambda_i * x_i)$, \\
Причем равенство достигается тогда и только тогда, когда $x_1 = \; ... \; = x_n$.\\
Тогда если взять $x_i = Pr(A_i)$ и $\lambda_i = \frac{1}{k}$ $\forall i \in 1..k$, получаем:\\
$\sum\limits_{i = 1}^k \frac{1}{k}(-Pr(A_i) * \log Pr(A_i)) \leq -\sum\limits_{i = 1}^k \frac{1}{k}Pr(A_i) * \log (\sum\limits_{i = 1}^k \frac{1}{k}Pr(A_i))$\\
$-\frac{1}{k} \sum\limits_{i = 1}^k Pr(A_i) * \log Pr(A_i) \leq -\frac{1}{k} \log\frac{1}{k}$\\
$-\sum\limits_{i = 1}^k Pr(A_i) * \log Pr(A_i) \leq \log k$\\
\item Из предыдущего пункта и условия равенства для неравенства Йенсена следует, что максимально возможную энтропию для КСС с k исходами, равную $\log k$, имеет схема с k равновероятными исходами.\\
\item $H(\alpha) = 0 \Leftrightarrow \exists!$ достоверный исход в $\alpha$. \\
Доказательство в обе стороны очевидно. Если в сумме, составляющей энтропию, есть хоть одно ненулевое слагаемое, то и энтропия ненулевая, а $x \log x = 0$ только при $x = 0$ и $x = 1$.
\end{itemize}
\subsubsection{Энтропия пересечения и условная энтропия}
Пусть есть две КСС - $\alpha$ с исходами $A_1, \; ... \; , A_k$ и $\beta$ с исходами $B_1, \; ... \; , B_l$. Их пересечением $\alpha \cap \beta$ называют схему, исходы которой - $A_i \cap B_j$ $\forall i \in 1..k, \forall j \in 1..l$.\\
Тогда $H(\alpha \cap \beta) = -\sum\limits_{i = 1}^k \sum\limits_{j = 1}^l Pr(A_i \cap B_j) * \log Pr(A_i \cap B_j)$\\
Воспользовавшись тем, что $Pr(A_i \cap B_j) = Pr(A_i) * Pr(B_j | A_i)$, получаем\\
$H(\alpha \cap \beta) = -\sum\limits_{i = 1}^k \sum\limits_{j = 1}^l Pr(A_i) * Pr(B_j | A_i) * (\log Pr(A_i) + \log Pr(B_j | A_i)) =$ \\ 
$ = -\sum\limits_{i = 1}^k \sum\limits_{j = 1}^l Pr(A_i) * Pr(B_j | A_i) * \log Pr(A_i) - \sum\limits_{i = 1}^k \sum\limits_{j = 1}^l Pr(A_i) * Pr(B_j | A_i) * \log Pr(B_j | A_i) = $\\
$ = -\sum\limits_{i = 1}^k Pr(A_i) * \log Pr(A_i) * \sum\limits_{j = 1}^l Pr(B_j | A_i) + \sum\limits_{i = 1}^k  Pr(A_i) * (-\sum\limits_{j = 1}^l Pr(B_j | A_i) * \log Pr(B_j | A_i)) = $\\
$ = -\sum\limits_{i = 1}^k Pr(A_i) * \log Pr(A_i) + \sum\limits_{i = 1}^k  Pr(A_i) * (-\sum\limits_{j = 1}^l Pr(B_j | A_i) * \log Pr(B_j | A_i)) = $\\
$ = H(\alpha) + \sum\limits_{i = 1}^k  Pr(A_i) * (-\sum\limits_{j = 1}^l Pr(B_j | A_i) * \log Pr(B_j | A_i))$.\\
$H(\beta | A_i) = -\sum\limits_{j = 1}^l Pr(B_j | A_i) * \log Pr(B_j | A_i)$ называют условной энтропией $\beta$ при условии $A_i$.\\
$H_{\alpha}(\beta) = \sum\limits_{i = 1}^k  Pr(A_i) * H(\beta | A_i)$ называют (средней) условной энтропией $\beta$ при условии $\alpha$.\\
Таким образом, в итоге формулу можно записать как \\
$H(\alpha \cap \beta) = H(\alpha) + H_{\alpha}(\beta)$.\\
Докажем, что  $0 \leq H_{\alpha}(\beta) \leq H(\beta)$:\\
Левая часть очевидна из тех же соображений, из которых следует неотрицательность энтропии.\\
чтобы доказать правую, напишем для зафиксированного j, функции $f(x) = -x * \log x$, $\lambda_i = Pr(A_i)$, $x_i = Pr(B_j | A_i)$ $\forall i \in 1..k$ неравенство Йенсена:\\
$\sum\limits_{i = 1}^k Pr(A_i) * (- Pr(B_j | A_i) * \log Pr(B_j | A_i)) \leq -(\sum\limits_{i = 1}^k Pr(A_i) * Pr(B_j | A_i)) * \log \sum\limits_{i = 1}^k Pr(A_i) * Pr(B_j | A_i)$\\
Преобразуем правую часть:\\
$-(\sum\limits_{i = 1}^k Pr(A_i) * Pr(B_j | A_i)) * \log \sum\limits_{i = 1}^k Pr(A_i) * Pr(B_j | A_i) = -(\sum\limits_{i = 1}^k Pr(B_j \cap A_i)) * \log \sum\limits_{i = 1}^k Pr(B_j \cap A_i) = -Pr(B_j) * \log Pr(B_j)$\\
После чего просуммируем обе части неравенства по j:\\
$\sum\limits_{j = 1}^l \sum\limits_{i = 1}^k Pr(A_i) * (- Pr(B_j | A_i) * \log Pr(B_j | A_i)) \leq \sum\limits_{j = 1}^l (-Pr(B_j) * \log Pr(B_j))$\\
$\sum\limits_{i = 1}^k Pr(A_i) * \sum\limits_{j = 1}^l (- Pr(B_j | A_i) * \log Pr(B_j | A_i)) \leq -\sum\limits_{j = 1}^l Pr(B_j) * \log Pr(B_j)$\\
$\sum\limits_{i = 1}^k Pr(A_i) * H(\beta | A_i) \leq H(\beta)$\\
$H_{\alpha}(\beta) \leq H(\beta)$.\\
Из условия равенства для неравенства Йенсена следует, что $H_{\alpha}(\beta) = H(\beta) \Leftrightarrow$ все $Pr(B_j | A_i)$ равны между собой.\\
По формуле полной вероятности,\\
$\forall j \in 1..l \; Pr(B_j) = \sum\limits_{i = 1}^k Pr(B_j | A_i) * Pr(A_i)$\\
Используя равенство всех $Pr(B_j | A_i)$ пишем:\\
$\forall j \in 1..l \; Pr(B_j) = Pr(B_j | A_1) * \sum\limits_{i = 1}^k Pr(A_i)$\\
$\forall j \in 1..l \; Pr(B_j) = Pr(B_j | A_1)$\\
То есть, $\forall i \in 1..k, \forall j \in 1..l \; Pr(B_j) = Pr(B_j | A_i)$
Вспомним теперь определение взаимно независимых событий:\\
A и B независимы $\Leftrightarrow Pr(A \cap B) = Pr(A) * Pr(B) \Leftrightarrow Pr(A) * Pr(B | A) = Pr(A) * Pr(B) \Leftrightarrow Pr(B | A) = Pr(B)$.\\
КСС $\alpha$ и $\beta$ называются независимыми, когда все исходы $\alpha$ независимы со всеми исходами $\beta$.\\
В таком случае, $H_{\alpha}(\beta)$ максимальна и равна $H(\beta)$.
\subsubsection{Количество информации}
Величина $I(\alpha, \beta) = H(\beta) - H_{\alpha}(\beta)$ называется количеством информации.\\
Запишем несколько свойств количества информации, которые доказываются простой проверкой:\\
\begin{itemize}
\item $I(\alpha, \beta) \geq 0$\\
\item $I(\alpha, \beta) = H(\beta) \Leftrightarrow H_{\alpha}(\beta) = 0$\\
\item $I(\alpha, \beta) = I(\beta, \alpha)$\\
\item $I(\alpha, \beta) = 0 \Leftrightarrow$ $\alpha$ и $\beta$ независимы.
\end{itemize}
Пример:\\
Загадано натуральное число $x \in 1..N$\\
$\beta$ - опыт, состоящий в нахождении x, $\beta_m$ - опыт, показывающий, делится ли x на m, $m \in 1..N$.\\
У $\beta$ N исходов, у $\beta_m$ два исхода.\\
$H_{\beta_m}(\beta) = Pr\{x \vdots m\} * H(\beta | "x \vdots m") + Pr\{x \not\vdots m\} * H(\beta | "x \not\vdots m")$\\
Если обозначить количество чисел от 1 до N, которые делятся на m, как $q = \lfloor \frac{N}{m} \rfloor$, то мы получим:\\
$Pr\{x \vdots m\} = \frac{q}{N}$\\
$Pr\{x \not\vdots m\} = \frac{N - q}{N}$\\
$H(\beta | "x \vdots m") = -\sum\limits_{i \vdots m, i \in 1..N} \frac{1}{q} * \log \frac{1}{q} = -\frac{q}{q} * \log \frac{1}{q} = \log q$\\
Аналогично $H(\beta | "x \not\vdots m") = \log (N - q)$\\
Таким образом, $H_{\beta_m}(\beta) = \frac{q}{N} * \log q + \frac{N - q}{N} * \log (N - q)$\\
$I(\beta_m, \beta) = \log N - \frac{q}{N} * \log q - \frac{N - q}{N} * \log (N - q) = $\\
$ = \frac{q}{N} * \log N - \frac{q}{N} * \log q + \frac{N - q}{N} * \log N - \frac{N - q}{N} * \log (N - q) = $\\
$ = -\frac{q}{N} * \log \frac{q}{N} - \frac{N - q}{N} * \log \frac{N - q}{N} \leq \log 2$\\
Равенство достигается при $q = N - q = \frac{N}{2}$, то есть если N четно и $m = 2$.