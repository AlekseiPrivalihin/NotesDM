\subsection{12.04.19}
\subsubsection{Постановка задачи о назначениях.}
Пусть на двудольном графе G задана функция веса $l: \; E(G) \rightarrow \mathbb{N}_0$.\\
Заведем матрицу L размера $n \times n$ такую, что $\forall e \in E(G) = ij \; l(e) = L[i,j]$.\\
Весом паросочетания M будем называть сумму весов ребер, входящих в это паросочетание:\\
$w(M) = \sum\limits_{e \in M}l(e)$.\\
Задача о назначениях - построение максимального (оно же наибольшее, если граф полный) паросочетания минимального веса.\\
Мы решаем задачу о назначениях для полного двудольного графа с долями равной мощности.\\
Формально: пусть G - полный двудольный граф с долями I и J, $|I| = |J| = n$.\\
Нужно построить такую биекцию $\phi: \; I \rightarrow J$, что $l(\phi) = l(M)$ минимален, где $M = \{e = (i,\phi(i)) \; | \; i \in I\}$.\\
\subsubsection{Допустимые преобразования.}
Лемма 1: При добавлении к строке матрицы L числа $\alpha$ вес любого максимального паросочетания увеличится на $\alpha$.\\
Формально: $\forall i_0 \in I, alpha \in \mathbb{Z}$ пусть $L'[i, j] = L[i, j] + \alpha$, если $i = i_0$ и $L'[i,j] = L[i, j]$ иначе. Тогда $\forall \phi: \; I \rightarrow J$ - биекция: $l'(\phi) = l(\phi) + \alpha$.\\
Доказательство: в каждом максимальном паросочетании ровно одно ребро выходит из вершины $i_0$, и его вес увеличился на $\alpha$, а веса ребер, выходящих из других вершин, не изменились.\\
Лемма 2: аналогично для столбцов.\\
Преобразования матрицы L, описанные в леммах 1 и 2, будем называть допустимыми.\\
Замечание: пусть $L'$ получена из $L$ конечной последовательностью допустимых преобразований.\\
Тогда если $\phi_1, \phi_2$ - максимальные паросочетания, $l(\phi_1) \leq l(\phi_2) \Leftrightarrow l'(\phi_1) \leq l'(\phi_2)$.\\
То есть, допустимые преобразования сохраняют упорядоченность максимальных паросочетаний по весу, в частности, это значит, что решение задачи о назначениях для $L'$ будет также решением для $L$.\\
\subsubsection{Венгерский метод решения задачи о назначениях.}
Кун, Кениг, Эгервари.\\
Шаг 1: сначала в каждой строке найдем минимум и вычтем его из этой строки. Затем сделаем то же самое для столбцов.\\
Получим матрицу $L_0$, в каждой строке и каждом столбце которой есть хотя бы один ноль.\\
Шаг 2: рассмотрим $G_0 \leq G$ такой, что $G_0 = (V(G), E_0 = \{e \in E(G) \; | \; l_0(e) = 0\})$\\
$\phi_0$ - наибольшее паросочетание на $G_0$.\\
$\phi_0$ задает $M_0 \subseteq E(G)$ и $C_0$ - контролирующее множество для $G_0$, $|M_0| = |C_0|$.\\
Если $|M_0| = n$, то мы нашли решение задачи о назначениях, т.к. паросочетание будет наибольшим в $G$ и $l_0(\phi_0) = 0$.\\
Иначе $|M_0| < m$. Тогда $C_0 = I_0 \cup J_0$, $I_0 \subseteq I$, $J_0 \subseteq J$.\\
$\delta = \min\limits_{i \not\in I_0, j \not\in J_0} L_0[i,j]$ - вес минимального ребра, не контролируемого $C_0$, а не минимальный ненулевой вес ребра в $L_0$!\\
Теперь $\forall i \not\in I_0$ $L_0'[i, j] = L_0[i,j] - \delta \; \forall j \in J$\\
$\forall j \in J_0$ $L_1[i, j] = L_0'[i,j] + \delta \; \forall i \in I$\\
То есть, вычли $\delta$ из всех неподконтрольных строк, а потом прибавили ко всем подконтрольным столбцам.\\
$\forall \psi$ - наибольшее паросочетание в G, $l_1(\psi) = l_0(\psi) - \delta(n - |I_0|) + \delta|J_0| = l_0(\psi) - \delta(n - (|I_0| + |J_0|) = l_0(\psi) - \delta(n - |C_0|) < l_0(\psi)$, так как $|C_0| < n$.\\
Теперь возвращаемся к шагу 2, уже с $L_1$ вместо $L_0$.\\
Алгоритм завершится, так как вес нашего паросочетания постоянно уменьшается на целое число, но при этом не может быть меньше 0.